\section{Methodology}\label{sec:metodo}

\subsection{Forward problem}

Let $\mathbf{d}^{o}$ be the observed data vector, whose $i$th element $d^{o}_{i}$, $i = 1, \dots, N$, is the total-field anomaly produced by a 3-D source (Fig. \ref{fig:obs}a) at the point $(x_{i}, y_{i}, z_{i})$ of a Cartesian coordinate system with $x$-, $y$- and $z$- axes pointing to north, east and down, respectively. We assume that the direction of the total magnetization vector of the source is constant and known. We approximate the volume of the source by a set of $L$ vertically juxtaposed 3-D prisms (Fig. \ref{fig:obs}b) by following the same approach of \cite{oliveirajr-etal2011} and \cite{oliveirajr-barbosa2013}. The depth to the top of the shallowest prism is defined by $z_{0}$ and $m_{0}$ is the constant total-magnetization intensity of all prisms. The horizontal cross-section of each prism (Fig. \ref{fig:prism_parameters}) 
is described by a polygon with a fixed number $V$ of vertices equally spaced from $0^{\circ}$ to $360^{\circ}$, which are described in polar coordinates referred to an internal origin $O^{k}$. The radii of the vertices ($r^{k}_{j}$, $j=1,\dots , V$, $k=1,\dots ,L$), the horizontal coordinates ($x_{0}^{k}$ and $y_{0}^{k}$, $k=1,\dots ,L$) of the origins $O^{k}$, $k=1,\dots ,L$, and the thickness $dz$ of the $L$ vertically stacked prisms (Fig. \ref{fig:obs}b) are arranged in a $M \times 1$ parameter vector $\mathbf{p}$, $M = L (V + 2) + 1$, given by
\begin{equation}
\mathbf{p} = \left[ \begin{array}{@{}*{12}{c}@{}}
{\mathbf{r}^{1}}^{\mathsf{T}} & x_{0}^{1} & y_{0}^{1} & \dots & {\mathbf{r}^{L}}^{\mathsf{T}} & x_{0}^{L} & y_{0}^{L} & dz \\
\end{array} \right]^{\mathsf{T}} \: ,
\label{eq:p-vector}
\end{equation}
where ``$^{\mathsf{T}}$" denotes transposition and $\mathbf{r}^{k}$ is a $V \times 1$ vector containing the radii $r^{k}_{j}$ 
of the $k$th prism.
Let $\mathbf{d} (\mathbf{p}, m_{0}, z_{0})$ be the predicted data vector, 
whose $i$th element 
\begin{equation}
d_{i} (\mathbf{p}, m_{0}, z_{0}) \equiv m_{0} \: \sum\limits_{k=1}^{L} f_{i}^{k}(\mathbf{r}^{k}, x_{0}^{k}, y_{0}^{k}, dz, z_{0}), \quad i = 1, \dots, N \: ,
\label{eq:predicted-data-i}
\end{equation}
is the total-field anomaly produced by the ensemble of $L$ prisms at the $i$th observation point ($x_{i}, y_{i}, z_{i}$). In Eq. \ref{eq:predicted-data-i}, $f_{i}^{k}(\mathbf{r}^{k}, x_{0}^{k}, y_{0}^{k}, dz, z_{0})$ is the total-field anomaly produced, at the observation point ($x_{i}, y_{i}, z_{i}$), by the $k$th prism with unitary magnetization intensity and depth to the top $z_{1}^{k} = z_{0} + (k-1)dz$. We calculate $d_{i} (\mathbf{p}, m_{0}, z_{0})$ (Eq. \ref{eq:predicted-data-i}) by using the Python package Fatiando a Terra \cite[]{uieda-etal2013}, which implements the formulas proposed by \cite{plouff1976}.

\subsection{Inverse problem for estimating $\mathbf{p}$}

Given the depth to the top of shallowest prism $z_{0}$ and the total-magnetization
intensity $m_{0}$ of all prisms, we solve a constrained nonlinear problem to estimate the parameter vector $\mathbf{p}$ (Eq. \ref{eq:p-vector}) by minimizing 
the goal function
\begin{equation}
\Gamma (\mathbf{p}, m_{0}, z_{0}) = \phi (\mathbf{p}, m_{0}, z_{0}) + 
\sum\limits^{7}_{\ell =1} \alpha_{\ell} \, \varphi_{\ell}(\mathbf{p}) \: ,
\label{eq:gamma}
\end{equation}
subject to the inequality constraints
\begin{equation}
p_{l}^{min} < p_{l} < p_{l}^{max}, \quad l = 1, \dots, M \: .
\label{eq:inequality-constraint}
\end{equation}
The first term in the right side of Eq. \ref{eq:gamma} is the data-misfit 
function given by
\begin{equation}\label{eq:misfit}
\phi (\mathbf{p}, m_{0}, z_{0}) = \frac{1}{N} \| \mathbf{d}^{o} - 
\mathbf{d}(\mathbf{p}, m_{0}, z_{0}) \|_{2}^{2} \: ,
\end{equation}
which represents the normalized squared Euclidean norm of the difference between the 
observed $\mathbf{d}^{o}$ and predicted $\mathbf{d}(\mathbf{p}, m_{0}, z_{0})$ data
vector, whose $i$th element is the predicted data $d_{i} (\mathbf{p}, m_{0}, z_{0})$ 
(Eq. \ref{eq:predicted-data-i}).
The second term in the right side of Eq. \ref{eq:gamma} represents the 
weighted sum of the seven constraint functions $\varphi_{\ell}(\mathbf{p}), \: 
\ell = 1, \dots, 7$, described in the following section \textit{Constraint functions}.
In Eq. \ref{eq:gamma}, $\alpha_{\ell}$ is a positive number representing 
the weight of the $\ell$th constraint function $\varphi_{\ell}(\mathbf{p})$.
In the inequality constraints (Eq. \ref{eq:inequality-constraint}), 
$p_{l}^{min}$ and $p_{l}^{max}$ are, respectively, the lower and upper limits for 
the $l$th element $p_{l}$ of the parameter vector $\mathbf{p}$ 
(Eq. \ref{eq:p-vector}). 
These limits are defined by the interpreter based on the knowledge about the 
horizontal and total depth extensions of the magnetic source. 
Details about how we set the weights $\alpha_{\ell}$ and the limits in the
inequality constraints are presented in the section 
\textit{Computational procedures} later in this article.

To solve our nonlinear inverse problem we use a gradient-based method and,
consequently, we need to define an initial approximation $\hat{\mathbf{p}}_{(0)}$ 
for the parameter vector $ \mathbf{p} $ (Eq. \ref{eq:p-vector}). 
Then our method iteratively updates this initial approximation to obtain 
an estimated parameter vector $\hat{\mathbf{p}}_{(f)}$ minimizing the goal function
$\Gamma (\mathbf{p}, m_{0}, z_{0})$ (Eq. \ref{eq:gamma}),
for given values of total-magnetization intensity $m_{0}$ and depth to the top
of the shallowest prism $z_{0}$. 
Here, we use the superscript hat ``~$\hat{}$~" to denote initial approximation or
estimated parameter vector.

Since we are using a gradient-based method, we need to define the 
gradient vector $\boldsymbol{\nabla}\Gamma(\mathbf{p})$ and Hessian matrix 
$\mathbf{H}(\mathbf{p})$ of the goal function $\Gamma(\mathbf{p})$ 
(Eq. \ref{eq:gamma}), both of them computed with respect to the parameter vector
$\mathbf{p}$. That is why we define them by omitting the parameters $m_{0}$ and
$z_{0}$. They are given by:
\begin{equation}\label{eq:gamma_gradient}
\boldsymbol{\nabla}\Gamma (\mathbf{p}) = \boldsymbol{\nabla}\phi (\mathbf{p}) + 
\sum\limits^{7}_{\ell =1} \alpha_{\ell} \, \boldsymbol{\nabla}\varphi_{\ell}(\mathbf{p})
\end{equation}
and
\begin{equation}\label{eq:gamma_hessian}
\mathbf{H} (\mathbf{p}) = \mathbf{H}_\phi (\mathbf{p}) + \sum\limits^{7}_{\ell =1} \alpha_{\ell} \, \mathbf{H}_\ell \: ,
\end{equation}
where the gradient vector and the Hessian matrix of the misfit function 
$\phi(\mathbf{p})$ (Eq. \ref{eq:gamma}) are respectively given by:
\begin{equation}\label{eq:phi_gradient}
\boldsymbol{\nabla} \phi(\mathbf{p}) = - \frac{2}{N}\mathbf{G}(\mathbf{p})^{\mathsf{T}}[\mathbf{d}^o - 
\mathbf{d}(\mathbf{p}, m_{0}, z_{0})]
\end{equation} 
and 
\begin{equation}\label{eq:phi_hessian}
\mathbf{H}_{\phi}(\mathbf{p}) = \frac{2}{N}\mathbf{G}(\mathbf{p})^{\mathsf{T}}\mathbf{G}(\mathbf{p}) \: .
\end{equation}
In Eqs \ref{eq:gamma_gradient} and \ref{eq:gamma_hessian}, the terms 
$\boldsymbol{\nabla} \varphi_{\ell}(\mathbf{p})$ and $\mathbf{H}_{\ell}$, 
$\ell = 1, \dots, 7$, are the gradient vectors and Hessian matrices of the constraint functions, respectively. In Eqs \ref{eq:phi_gradient} and \ref{eq:phi_hessian},
$\mathbf{G}(\mathbf{p})$ is an $N \times M$ matrix whose element $ij$ is the derivative of the predicted data $d_{i}(\mathbf{p}, m_{0}, z_{0})$ (Eq. \ref{eq:predicted-data-i}) with respect to the $j$ element $p_{j}$ of 
the parameter vector $\mathbf{p}$ (Eq. \ref{eq:p-vector}). Details about the constraint functions $\varphi_\ell(\mathbf{p})$, $\ell = 1, \dots, 7$, as well as the numerical procedure to solve this nonlinear inverse problem are given in the following sections.

\subsection{Constraint functions}\label{sec:constraints}

To explain the constraint functions $\varphi_{\ell}(\mathbf{p})$ (Eq. \ref{eq:gamma}), $\ell = 1, \dots, 7$, used here to obtain stable solutions and introduce prior information about the magnetic source, we have organized them into the following three groups.

\subsubsection{Smoothness constraints}

This group is formed by variations of the first-order Tikhonov regularization \cite[][ p. 103]{aster-etal2019} that imposes smoothness on the radii $r_{j}^{k}$ and the Cartesian coordinates $x_{0}^{k}$ and $y_{0}^{k}$ of the origin $O^{k}$, $j = 1, \dots, V$, $k = 1, \dots, L$, defining the horizontal section of each prism (Fig.\ref{fig:obs}b).
They were proposed by \cite{oliveirajr-etal2011} and \cite{oliveirajr-barbosa2013} and play a very important role in introducing prior information about the shape of the source. 

The first constraint of this group is the \textit{smoothness constraint on the adjacent radii defining the horizontal section of each prism}. This constraint imposes that adjacent radii $r_{j}^{k}$ and $r_{j+1}^{k}$ within each prism must be close to each other. It forces the estimated prism to be approximately cylindrical. Mathematically, the constraint is given by \begin{equation}\label{eq:phi1}
\begin{split}
\varphi_{1}(\mathbf{p}) &= \sum\limits^{L}_{k=1}\left[\left(r^{k}_{V}-r^{k}_{1}\right)^2 + \sum\limits^{V-1}_{j=1}\left(r^{k}_{j}-r^{k}_{j+1}\right)^2\right]\\
 &= \mathbf{p}^{\mathsf{T}} \mathbf{R}^{\mathsf{T}}_{1}\mathbf{R}_{1} \mathbf{p} \quad ,
\end{split}
\end{equation}
where
\begin{equation}
\mathbf{R}_{1} = 
\begin{bmatrix}
\mathbf{S}_{1} &
\mathbf{0}_{LV \times 1} \\
\end{bmatrix}_{LV \times M} \quad ,
\label{eq:R1-matrix}
\end{equation}
\begin{equation}
\mathbf{S}_{1} = 
\mathbf{I}_{L} \otimes 
\begin{bmatrix}
\left( \mathbf{I}_{V} - \mathbf{D}_{V}^\mathsf{T} \right) & \mathbf{0}_{V \times 2} \\
\end{bmatrix}_{V \times (V+2)} \quad ,
\label{eq:S1-matrix}
\end{equation}
$ \mathbf{0}_{LV \times 1} $ is an $ LV \times 1 $ vector with null elements,
$\mathbf{I}_{L}$ is the identity matrix of order $L$, ``$\otimes$" denotes the Kronecker product \cite[][ p. 243]{horn_johnson1991}, $\mathbf{0}_{V \times 2}$ is a $V \times 2$ matrix with null elements, 
$\mathbf{I}_{V}$ is the identity matrix of order $V$ and $\mathbf{D}_{V}^\mathsf{T}$ is the upshift permutation matrix of order $V$ \cite[][ p. 20]{golub-vanloan2013}. The gradient and Hessian of function $\varphi_{1}(\mathbf{p})$ (Eq. \ref{eq:phi1}) are given by:
\begin{equation}\label{eq:phi1_grad}
\boldsymbol{\nabla}\varphi_{1}(\mathbf{p}) = 2 \mathbf{R}^\mathsf{T}_{1}\mathbf{R}_{1}\mathbf{p} \quad ,
\end{equation}
and
\begin{equation}\label{eq:phi1_hessian}
\mathbf{H}_{1}(\mathbf{p}) = 2\mathbf{R}^\mathsf{T}_{1}\mathbf{R}_{1} \quad .
\end{equation}

The second constraint of this group is the \textit{smoothness constraint on the adjacent radii of the vertically adjacent prisms}, which imposes that adjacent radii $r_{j}^{k}$ and $r_{j}^{k+1}$ within vertically adjacent prisms must be close to each other. This constraint forces the shape of all prisms to be similar to each other
and is given by
\begin{equation}\label{eq:phi2}
\begin{split}
\varphi_{2}(\mathbf{p}) &= \sum\limits^{L-1}_{k=1}\left[\sum\limits^{V}_{j=1}\left(r^{k+1}_{j}-r^{k}_{j}\right)^2\right] \\
&= \mathbf{p}^{\mathsf{T}} \mathbf{R}^{\mathsf{T}}_{2}\mathbf{R}_{2}\mathbf{p}
\end{split} \quad ,
\end{equation}
where 
\begin{equation}
\mathbf{R}_{2} = 
\begin{bmatrix}
\mathbf{S}_{2} & \mathbf{0}_{(L-1)V \times 1} \\
\end{bmatrix}_{(L-1)V \times M} \quad ,
\label{eq:R2-matrix}
\end{equation}
\begin{equation}
\mathbf{S}_{2} =
\left( 
\begin{bmatrix} \mathbf{I}_{L-1} & \mathbf{0}_{(L-1) \times 1} \end{bmatrix} -
\begin{bmatrix} \mathbf{0}_{(L-1) \times 1} & \mathbf{I}_{L-1} \end{bmatrix} 
\right) \otimes 
\begin{bmatrix} \mathbf{I}_{V} & \mathbf{0}_{V \times 2} \end{bmatrix} \quad ,
\label{eq:S2-matrix}
\end{equation}
$\mathbf{0}_{(L-1)V \times 1}$ is an $(L-1)V \times 1$ vector with null elements,
$\mathbf{0}_{(L-1) \times 1}$ is an $(L-1) \times 1$ vector with null elements and 
$\mathbf{I}_{L-1}$ is the identity matrix of order $L-1$. The gradient and Hessian of function $\varphi_{2}(\mathbf{p})$ (Eq. \ref{eq:phi2}) are given by:
\begin{equation}\label{eq:phi2_grad}
\boldsymbol{\nabla}\varphi_{2}(\mathbf{p}) = 2\mathbf{R}^\mathsf{T}_{2}\mathbf{R}_{2}\mathbf{p} \quad ,
\end{equation}
and
\begin{equation}\label{eq:phi2_hessian}
\mathbf{H}_{2}(\mathbf{p}) = 2\mathbf{R}^\mathsf{T}_{2}\mathbf{R}_{2} \quad .
\end{equation}

The last constraint of this group is the \textit{smoothness constraint on the horizontal position of 
the arbitrary origins of the vertically adjacent prisms}. This constraint imposes that the estimated horizontal 
Cartesian coordinates $(x_{0}^{k}, y_{0}^{k})$ and $(x_{0}^{k+1}, y_{0}^{k+1})$ of the origins $O^{k}$ and $O^{k+1}$ 
of adjacent prisms must be close to each other. It forces the centers of the prisms to be vertically aligned. This constraint 
is given by
\begin{equation}\label{eq:phi3}
\begin{split}
\varphi_{3}(\mathbf{p}) &= \sum\limits^{L-1}_{k=1}\left[\left(x_{0}^{k+1} - x_{0}^{k}\right)^2 + \left(y_{0}^{k+1} - y_{0}^{k}\right)^2 \right] \\
&= \mathbf{p}^{\mathsf{T}} \mathbf{R}^{\mathsf{T}}_{3}\mathbf{R}_{3}\mathbf{p}
\end{split} \quad ,
\end{equation}
where 
\begin{equation}
\mathbf{R}_{3} = 
\begin{bmatrix}
\mathbf{S}_{3} & \mathbf{0}_{(L-1)2 \times 1} \\
\end{bmatrix}_{(L-1)2 \times M} \quad ,
\label{eq:R3-matrix}
\end{equation}
\begin{equation}
\mathbf{S}_{3} =
\left( 
\begin{bmatrix} \mathbf{I}_{L-1} & \mathbf{0}_{(L-1) \times 1} \end{bmatrix} -
\begin{bmatrix} \mathbf{0}_{(L-1) \times 1} & \mathbf{I}_{L-1} \end{bmatrix} 
\right) \otimes 
\begin{bmatrix} \mathbf{0}_{2 \times V} & \mathbf{I}_{2} \end{bmatrix} \quad ,
\label{eq:S3-matrix}
\end{equation}
$\mathbf{0}_{(L-1)2 \times 1}$ is an $(L-1)2 \times 1$ vector with null elements,
$\mathbf{0}_{2 \times V}$ is a $2 \times V$ matrix with null elements and 
$\mathbf{I}_{2}$ is the identity matrix of order $2$. The gradient and Hessian of function $\varphi_{3}(\mathbf{p})$ (Eq. \ref{eq:phi3}) are given by:
\begin{equation}\label{eq:phi3_grad}
\boldsymbol{\nabla}\varphi_{3}(\mathbf{p}) = 2 \mathbf{R}^{\mathsf{T}}_{3}\mathbf{R}_{3}\mathbf{p} \quad ,
\end{equation}
and
\begin{equation}\label{eq:phi3_hessian}
\mathbf{H}_{3}(\mathbf{p}) = 2 \mathbf{R}^{\mathsf{T}}_{3}\mathbf{R}_{3} \quad .
\end{equation}


\subsubsection{Equality constraints}


This group is formed by two constraints that were proposed by \cite{oliveirajr-etal2011} and \cite{oliveirajr-barbosa2013} by following the same approach proposed \cite{barbosa-etal1997} and \cite{barbosa-1999a}. 
They introduce a priori information about the shallowest prism and are suitable for outcropping sources.

The \textit{source’s outcrop constraint} imposes that the horizontal cross-section of the shallowest prism must be close to known outcropping boundary of the geologic source.
The horizontal cross-section of the known outcropping boundary separating the geologic source from the host rock is described by the radii $\tilde{r}_{1}^{0} \dots \tilde{r}_{V}^{0}$. Mathematically, this constraint is given by
\begin{equation}\label{eq:phi4}
\begin{split}
\varphi_{4}(\mathbf{p}) &= \left[ \sum\limits^{V}_{j=1}\left(r^{1}_{j}-\tilde{r}^{0}_{j}\right)^2\right] \\
&= \left(\mathbf{R}_{4} \mathbf{p} - \mathbf{a} \right)^{\mathsf{T}} 
\left(\mathbf{R}_{4} \mathbf{p} - \mathbf{a} \right)
\end{split} \quad ,
\end{equation}
where $\mathbf{a}$ is a $V \times 1$ vector containing the radii of the polygon defining the outcropping boundary
\begin{equation}
\mathbf{a} = \left[ \begin{array}{@{}*{12}{c}@{}}
\tilde{r}_{1}^{0} & \dots & \tilde{r}_{V}^{0} \\
\end{array} \right]^{\mathsf{T}} \: ,
\label{eq:a-vector}
\end{equation}
and
\begin{equation}
\mathbf{R}_{4} = 
\begin{bmatrix}
\mathbf{I}_{V} & \mathbf{0}_{V \times (M-V)} \\
\end{bmatrix}_{V\times M},
\label{eq:R4-matrix}
\end{equation}
where $\mathbf{I}_{V}$ is the identity matrix of order $V$ and 
$\mathbf{0}_{V \times (M-V)}$ is a $V \times (M-V)$ matrix with null elements.
The gradient and Hessian of function $\varphi_{4}(\mathbf{p})$ (Eq. \ref{eq:phi4}) are given by:
\begin{equation}\label{eq:phi4_grad}
\boldsymbol{\nabla}\varphi_{4}(\mathbf{p}) = 2 \mathbf{R}_{4}^{\mathsf{T}} 
\left(\mathbf{R}_{4} \mathbf{p} - \mathbf{a} \right) \quad ,
\end{equation}
and
\begin{equation}\label{eq:phi4_hessian}
\mathbf{H}_{4}(\mathbf{p}) = 2 \mathbf{R}^{\mathsf{T}}_{4}\mathbf{R}_{4} \quad .
\end{equation}

The \textit{source's horizontal location constraint} imposes that the horizontal Cartesian coordinates of the origin within 
the shallowest prism must be as close as possible to a known outcropping point given by the horizontal Cartesian coordinates $(\tilde{x}_{0}^{0}, \: \tilde{y}_{0}^{0})$. 
This constraint is given by
\begin{equation}\label{eq:phi5}
\begin{split}
\varphi_{5}(\mathbf{p}) &= \left[\left(x_{0}^{1} - \tilde{x}_{0}^{0}\right)^2 
+ \left(y_{0}^{1} - \tilde{y}_{0}^{0}\right)^2\right] \\
&= \left(\mathbf{R}_{5} \mathbf{p} - \mathbf{b} \right)^{\mathsf{T}}
\left(\mathbf{R}_{5} \mathbf{p} - \mathbf{b}\right)
\end{split} \quad ,
\end{equation}
where $\mathbf{b}$ is a $2 \times 1$ vector containing the horizontal Cartesian coordinates of the outcropping point 
\begin{equation}
\mathbf{b} = \left[ \begin{array}{@{}*{12}{c}@{}}
\tilde{x}_{0}^{0} & \tilde{y}_{0}^{0} \\
\end{array} \right]^{\mathsf{T}} \: ,
\label{eq:b-vector}
\end{equation}
and
\begin{equation}
\mathbf{R}_{5} = 
\begin{bmatrix}
\mathbf{0}_{2 \times V} & \mathbf{I}_{2} & \mathbf{0}_{2 \times (M-V-2)} \\
\end{bmatrix}_{2 \times M},
\label{eq:R5-matrix}
\end{equation}
where $\mathbf{0}_{2 \times (M-V-2)}$ is a $2 \times (M-V-2)$ matrix 
with null elements. 
The gradient and Hessian of function $\varphi_{5}(\mathbf{p})$ (Eq. \ref{eq:phi5}) are given by:
\begin{equation}\label{eq:phi5_grad}
\boldsymbol{\nabla}\varphi_{5}(\mathbf{p}) = 2\mathbf{R}_{5}^{\mathsf{T}}
\left(\mathbf{R}_{5} \mathbf{p} - \mathbf{b}\right) \quad ,
\end{equation}
and
\begin{equation}\label{eq:phi5_hessian}
\mathbf{H}_{5}(\mathbf{p}) = 2 \mathbf{R}^{\mathsf{T}}_{5}\mathbf{R}_{5} \quad .
\end{equation}

\subsubsection{Minimum Euclidean norm constraints}

Two constraints use the zeroth-order Tikhonov regularization with the purpose of obtaining stable solutions without introducing prior information about the shape of the source. 
However, these two constraints combined with the interpretation model impose 
source compactness. 

The \textit{Minimum Euclidean norm of the radii} imposes that 
all estimated radii within each prism must be close to null values. This constraint was proposed by \cite{oliveirajr-etal2011} and \cite{oliveirajr-barbosa2013} and can be rewritten as follows
\begin{equation}\label{eq:phi6}
\begin{split}
\varphi_{6}(\mathbf{p}) &= \sum\limits^{L}_{k=1}\sum\limits^{V}_{j=1}\left(r_{j}^{k}\right)^2 \\
&= \mathbf{p}^{\mathsf{T}} \mathbf{R}_{6}^{\mathsf{T}} \mathbf{R}_{6} \mathbf{p}
\end{split} \quad ,
\end{equation}
where
\begin{equation}
\mathbf{R}_{6} = 
\begin{bmatrix}
\mathbf{S}_{6} & \mathbf{0}_{(M-1) \times 1} \\
\mathbf{0}_{1 \times (M-1)} & 0 \\
\end{bmatrix}_{M\times M} \quad ,
\label{eq:R6-matrix}
\end{equation}
and 
\begin{equation}
\mathbf{S}_{6} = 
\mathbf{I}_{L} \otimes 
\begin{bmatrix}
\mathbf{I}_{V}      & \mathbf{0}_{V \times 2} \\
\mathbf{0}_{2 \times V} & \mathbf{0}_{2 \times 2} \\
\end{bmatrix}_{(V+2) \times (V+2)} \quad ,
\label{eq:S6-matrix}
\end{equation}
where $\mathbf{0}_{2 \times 2}$ is a $2 \times 2$ matrix with null elements,
$\mathbf{0}_{V \times 2}$ is a $V \times 2$ matrix with null elements and
$\mathbf{0}_{2 \times V}$ is a $2 \times V$ matrix with null elements.
The gradient and Hessian of function $\varphi_{6}(\mathbf{p})$ (Eq. \ref{eq:phi6}) are given by:
\begin{equation}\label{eq:phi6_grad}
\boldsymbol{\nabla}\varphi_{6}(\mathbf{p}) = 2 \mathbf{R}_{6}^{\mathsf{T}} \mathbf{R}_{6} \mathbf{p} \quad ,
\end{equation}
and
\begin{equation}\label{eq:phi6_hessian}
\mathbf{H}_{6}(\mathbf{p}) = 2 \mathbf{R}^{\mathsf{T}}_{6}\mathbf{R}_{6} \quad .
\end{equation}

The other constraint, the \textit{Minimum Euclidean norm of the prism thickness}, imposes that the thickness of all prisms must be close to zero. We present this constraint to introduce a priori information about the maximum depth extent of the source which in turn is dependent on the depth to the top of the shallowest prism $z_{0}$. It is given by
\begin{equation}\label{eq:phi7}
\begin{split}
\varphi_{7}(\mathbf{p}) &= dz^2 \\
&= \mathbf{p}^{\mathsf{T}} \mathbf{R}_{7}^{\mathsf{T}} \mathbf{R}_{7} \mathbf{p}
\end{split} \quad ,
\end{equation}
where
\begin{equation}
\mathbf{R}_{7} =
\begin{bmatrix}
\mathbf{0}_{(M-1) \times (M-1)} & \mathbf{0}_{(M-1) \times 1} \\
\mathbf{0}_{1 \times (M-1)} & 1 \\
\end{bmatrix}_{ M \times M } \quad .
\end{equation}
The gradient and Hessian of function $\varphi_{7}(\mathbf{p})$ (Eq. \ref{eq:phi7}) are given by:
\begin{equation}\label{eq:phi7_grad}
\boldsymbol{\nabla}\varphi_{7}(\mathbf{p}) = 2 \mathbf{R}_{7}^{\mathsf{T}} \mathbf{R}_{7} \mathbf{p} \quad ,
\end{equation}
and
\begin{equation}\label{eq:phi7_hessian}
\mathbf{H}_{7}(\mathbf{p}) = 2 \mathbf{R}^{\mathsf{T}}_{7}\mathbf{R}_{7} \quad .
\end{equation}

\subsection{Computational procedures}

In this section, we present the computational procedures 
to solve the nonlinear inverse problem for estimating a parameter vector 
$\hat{\mathbf{p}}_{(f)}$ that minimizes the goal function 
$\Gamma(\mathbf{p}, m_{0}, z_{0})$ (Eq. \ref{eq:gamma}), subject to the inequality
constraints (Eq. \ref{eq:inequality-constraint}), for previously defined values of
depth to the top of the shallowest prism $z_{0}$ and total-magnetization intensity
$m_{0}$. We use an iterative gradient-based algorithm for solving the nonlinear
inverse problem.

This section is divided into four parts. The first part describes how to define  
the number of prisms $L$, the number of vertices $V$, total-magnetization 
direction and thickness $dz$ of all prisms, the initial approximation
$\hat{\mathbf{p}}_{(0)}$ for the parameter vector $ \mathbf{p} $ 
(Eq. \ref{eq:p-vector}), as well as the upper and lower limits in the inequality
constraints ($p_{l}^{min}$ and $p_{l}^{max}$ in Eq. \ref{eq:inequality-constraint}). 
All these variables are previously defined by the interpreter and never updated
in the nonlinear inverse problem.
The second part of this section explains how we define different 
tentative values for the depth to the top $z_{0}$ and total-magnetization
intensity $m_{0}$, as well as how we choose their optimum values.
In the third part we present the gradient-based algorithm for solving the nonlinear
inverse problem for a given tentative pair ($m_{0}$, $z_{0}$). 
Finally, we describe how to set the weights $ \alpha_{\ell} $ 
(Eq. \ref{eq:gamma}) in the fourth part of this section.

\subsubsection{Initial approximation $\hat{\mathbf{p}}_{(0)}$ and inequality constraints}

Our initial approximation is a uniformly-magnetized cylinder-like body formed by 
$L$ prisms, all of them with the same number of vertices $V$. The radii of the
vertices of all prisms ($r^{k}_{j}$, $j=1,\dots , V$, $k=1,\dots ,L$) have the 
same constant value, so that the initial approximation approaches a vertical
cylinder. 
The number of prisms $L$ and vertices $V$ are set based on the expected complexity 
of the magnetic source.
We start by computing the RTP anomaly, which has three purposes.
The first one is verifying if the used total-magnetization direction is valid.
The second purpose is defining the upper and lower limits of the inequality
constraints (Eq. \ref{eq:inequality-constraint}).
Finally, the third purpose of the RTP anomaly is defining the radius and horizontal
Cartesian coordinates of the center of the cylinder-like initial approximation.

It is well known that if the source has a uniform magnetization direction, 
the RTP anomaly is predominantly positive and decays to zero close to its 
horizontal boundaries \cite[e.g.,][p. 331]{blakely1996}. 
To compute this transformation, however, the interpreter must use declination and
inclination values close to those defining the true total-magnetization direction
of the source. 
Hence, the interpreter can validate the total-magnetization direction of the source 
by verifying if the computed RTP anomaly shows predominantly positive values.

By using the computed RTP anomaly, we also define the upper and lower limits 
($p_{l}^{min}$ and $p_{l}^{max}$ in Eq. \ref{eq:inequality-constraint}) for 
the parameters representing the radii $r^{k}_{j}$ of the vertices and the 
horizontal coordinates $x_{0}^{k}$ and $y_{0}^{k}$ of the origins $O^{k}$
of all prisms ($j=1,\dots , V$, $k=1,\dots ,L$). For the parameters representing
the radii $r^{k}_{j}$, the lower limit $p_{l}^{min}$ is close to zero and the 
upper limit $p_{l}^{max}$ is approximately defined by the radius of a circular area 
encompassing the region where the RTP anomaly is positive and decays to zero.
The lowermost and uppermost $x-$ and $y-$ coordinates of this circular area 
are used to define, respectively, the lower and upper limits for the 
horizontal coordinates $x_{0}^{k}$ and $y_{0}^{k}$. 

Next, we set the same thickness $dz$ for all prisms so that the resulting 
total thickness ($L \, dz$) is greater than that we expect for the true source.
The lower and upper limits ($p_{l}^{min}$ and $p_{l}^{max}$ in 
Eq. \ref{eq:inequality-constraint}) for the parameter $dz$ are set to be,
respectively, a value close to zero and a large value resulting in a 
total thickness ($L \, dz$) greater than that we expect for the true source.
At this point, we have fully defined the vector $\hat{\mathbf{p}}_{(0)}$ 
representing the cylinder-like initial approximation for the parameter 
vector $ \mathbf{p} $ (Eq. \ref{eq:p-vector}).

\subsubsection{Definition of optimum values for $m_{0}$ and $z_{0}$}

After defining the initial approximation $\hat{\mathbf{p}}_{(0)}$, we need
to set the depth to the top of the shallowest prism $z_0$ 
and the total-magnetization intensity $m_0$ for all prisms.
These two parameters are also previously defined by the interpreter and 
remain fixed along the iterations of our gradient-based algorithm for solving the 
nonlinear inverse problem. 

In the absence of a priori information, finding values close to the true ones
for $z_0$ and $m_0$ is a very difficult task due to the inherent ambiguity
in magnetic inversion. It is expected that estimated magnetic sources obtained by
using different combinations of $z_0$ and $m_0$ may produce similar data fits.
Because of that, we do not estimate their values in the nonlinear inverse problem.
Instead, we set ranges of tentative values for them.
For each tentative pair ($m_0$, $z_0$), we use the same previously defined 
cylinder-like initial approximation $\hat{\mathbf{p}}_{(0)}$ to 
obtain an independent estimated parameter vector $\hat{\mathbf{p}}_{(f)}$ 
minimizing the goal function $\Gamma (\mathbf{p}, m_{0}, z_{0})$ 
(Eq. \ref{eq:gamma}).
Note that this approach results in a set of estimated magnetic sources with
different depths to the top $z_0$, total-magnetization intensities $m_0$ and 
geometries defined by different estimated parameter vectors $\hat{\mathbf{p}}_{(f)}$.
Each estimated magnetic source results in a different value 
$\Gamma (\hat{\mathbf{p}}_{(f)}, m_{0}, z_{0})$ for the goal function.
On the discrete map of the goal function produced by the set of estimated magnetic
sources, each one associated with a tentative pair ($m_0$, $z_0$), 
we select the optimum values of $m_{0}$ and $z_{0}$ as those producing the smallest
value of $\Gamma (\hat{\mathbf{p}}_{(f)}, m_{0}, z_{0})$. 
We stress that all estimated parameter vectors $\hat{\mathbf{p}}_{(f)}$ are obtained
by using the same values for the weights $\alpha_{\ell}$ (Eq. \ref{eq:gamma}).
Details about how we set these weights are presented later, in the section
\textit{Setting the weights $\alpha_{1}-\alpha_{7}$}.

Usually, the range of $z_0$ includes the topography ($z_0 = 0$) and the range of 
$m_0$ is based on a priori information such as petrophysical studies.
In order to verify our initial approximation $\hat{\mathbf{p}}_{(0)}$ and the
ranges of tentative values for $z_0$ and $m_0$, we compute a discrete map
of the data-misfit function $\phi (\mathbf{p}, m_{0}, z_{0})$ (Eq. \ref{eq:misfit}).
All values in this map are produced by using the same previously defined 
initial approximation $\hat{\mathbf{p}}_{(0)}$. The only differences are the 
tentative values for $z_0$ and $m_0$. 
The computation of this discrete map is carried out, before performing the 
inversions, with the purpose of verifying if there is at least one point 
associated with a data-misfit function $\phi (\hat{\mathbf{p}}_{(0)}, m_{0}, z_{0})$ 
showing a relatively low value. In this case, there is at least one point on the
discrete map where the corresponding initial approximation $\hat{\mathbf{p}}_{(0)}$
and tentative values for $m_{0}$ and $z_{0}$ produce a predicted data vector 
$\mathbf{d}(\hat{\mathbf{p}}_{(0)}, m_{0}, z_{0})$ close to the observed 
data vector $\mathbf{d}^{o}$.
We stress that, at this step, we do not need a good data fit.
If we cannot identify any point on the discrete map of the data-misfit function 
$\phi (\hat{\mathbf{p}}_{(0)}, m_{0}, z_{0})$ associated with a satisfactory
data fit, we redefine the initial approximation $\hat{\mathbf{p}}_{(0)}$ 
and recompute the discrete map.

\subsubsection{Inversion algorithm}

We use the Levenberg-Marquardt algorithm \cite[e.g., ][ p. 240]{aster-etal2019} to solve the nonlinear inverse problem given by Eq. \ref{eq:gamma}.
The Levenberg-Marquardt algorithm is an iterative gradient-based method that, at each iteration $n$, updates the estimate parameter vector $\hat{\mathbf{p}}_{(n)}$ to obtain a new estimated parameter vector $\hat{\mathbf{p}}_{(n + 1)}$.
We compute this update by following the same strategy of \cite{barbosa-1999b}, \cite{oliveirajr-etal2011} and \cite{oliveirajr-barbosa2013} to incorporate the inequality constraints (Eq. \ref{eq:inequality-constraint}). 
This strategy consists in transforming each element $\hat{p}_{l} \in (p_{l}^{min}, p_{l}^{max})$ of the estimated parameter vector $\hat{\mathbf{p}}_{(n)}$ 
into the element $\hat{p}^{\dagger}_{l} \in (- \infty, + \infty)$ of an 
unconstrained vector $\hat{\mathbf{p}}^{\dagger}_{(n)}$ as follows:
\begin{equation}\label{eq:inequality-function}
\hat{p}^{\dagger}_{l} = -\ln\left(\frac{p_{l}^{max} - \hat{p}_{l}}{\hat{p}_{l} - p_{l}^{min}}\right) \: ,
\end{equation}
where $p_{l}^{min}$ and $p_{l}^{max}$ are defined in the inequality constraints 
(Eq. \ref{eq:inequality-constraint}).
The inverse transformation of each element $\hat{p}^{\dagger}_{l} \in (- \infty, + \infty)$ into the element $\hat{p}_{l} \in (p_{l}^{min}, p_{l}^{max})$ is the following:
\begin{equation}\label{eq:inv-inequality-function}
\hat{p}_{l} = p_{l}^{min} + \left(\frac{p_{l}^{max} - p_{l}^{min}}{ 1 + e^{-\hat{p}^{\dagger}_{l}} }\right) \: .
\end{equation}

At each iteration $n$ of our algorithm, a correction 
$\boldsymbol{\Delta}\hat{\mathbf{p}}^{\dagger}_{(n)}$ for the
unconstrained vector $\hat{\mathbf{p}}^{\dagger}_{(n)}$
is computed by solving the following linear system:
\begin{equation}\label{eq:linear-system}
\mathbf{Q}_{(n)}^{-1} \left[\mathbf{Q}_{(n)} \mathbf{H}^{\dagger}(\hat{\mathbf{p}}_{(n)}) \mathbf{Q}_{(n)} + \lambda_{(n)} \mathbf{I}_{M} \right] 
\mathbf{Q}_{(n)}^{-1} \boldsymbol{\Delta} \hat{\mathbf{p}}^{\dagger}_{(n)} 
= -\boldsymbol{\nabla}\Gamma(\hat{\mathbf{p}}_{(n)}) \: ,
\end{equation}
where $\lambda_{(n)}$ is a positive scalar (known as Marquardt parameter) which is adjusted at each iteration and is associated with the Levenberg-Marquardt method \cite[e.g., ][ p. 240]{silva-2001,aster-etal2019},
$\mathbf{I}_{M}$ is the identity matrix with order $M$, 
$\boldsymbol{\nabla}\Gamma(\hat{\mathbf{p}}_{(n)})$
is the gradient of the goal function (Eq. \ref{eq:gamma_gradient}), 
$\mathbf{H}^{\dagger}(\hat{\mathbf{p}}_{(n)})$ is a matrix given by
\begin{equation}\label{eq:H-dagger}
\mathbf{H}^{\dagger}(\hat{\mathbf{p}}_{(n)}) = \mathbf{H}(\hat{\mathbf{p}}_{(n)})\mathbf{T}(\hat{\mathbf{p}}_{(n)}) 
\end{equation}
and $\mathbf{Q}_{(n)}$ is a diagonal matrix proposed by \cite{marquardt_algorithm_1963} for scaling the parameter $\lambda_{(n)}$ 
at each iteration.
The element $ll$ of this diagonal matrix is given by
\begin{equation}\label{eq:Q-matrix}
q_{ll} = \frac{1}{\sqrt{h^{\dagger}_{ll}}} \: ,
\end{equation}
where $h^{\dagger}_{ll}$ is the element $ll$ of the matrix $\mathbf{H}^{\dagger}(\hat{\mathbf{p}}_{(n)})$ (Eq. \ref{eq:H-dagger}). 
In Eq. \ref{eq:H-dagger}, $\mathbf{H}(\hat{\mathbf{p}}_{(n)})$ is the Hessian matrix of the goal function (Eq. \ref{eq:gamma_hessian}) and $\mathbf{T}(\hat{\mathbf{p}}_{(n)})$ is a diagonal matrix whose element $ll$ is given by
\begin{equation}\label{eq:inequality-diag}
t_{ll} = \frac{(p_{l}^{max} - \hat{p}_{l} + \epsilon)(\hat{p}_{l} - p_{l}^{min} + \epsilon)}{p_{l}^{max} - p_{l}^{min}} \: ,
\end{equation}
where $\hat{p}_{l}$ is the $l$th element of the estimated parameter vector 
$\hat{\mathbf{p}}_{(n)}$ and $\epsilon$ is a small positive number 
($\approx 10^{-2}$) used to prevent null values.

After estimating $\boldsymbol{\Delta} \hat{\mathbf{p}}^{\dagger}_{(n)}$ by 
solving the linear system (Eq. \ref{eq:linear-system}), we update the unconstrained 
vector by computing
\begin{equation}\label{eq:p-k1-dagger}
\hat{\mathbf{p}}^{\dagger}_{(n + 1)} =
\hat{\mathbf{p}}^{\dagger}_{(n)} +
\boldsymbol{\Delta} \hat{\mathbf{p}}^{\dagger}_{(n)} \: .
\end{equation}
Next, we compute the elements of an updated parameter vector $\hat{\mathbf{p}}_{(n + 1)}$ by using Eq. \ref{eq:inv-inequality-function}.
Finally, we stop the iterative process by evaluating the invariance of the 
goal function (Eq. \ref{eq:gamma}) along successive iterations.
Specifically, we check if the inequality
\begin{equation}\label{eq:stop-criterion}
 \left| \frac{\Gamma (\hat{\mathbf{p}}_{(n +1)}) - 
 \Gamma (\hat{\mathbf{p}}_{(n)})}
 {\Gamma (\hat{\mathbf{p}}_{(n)})} 
 \right| \le \tau
\end{equation}
holds, where $\tau$ is a threshold value on the order of $10^{-3}$ to $10^{-4}$. 
Usually, the inversion algorithm converges at $8$-$12$ iterations.
In the evaluation of the goal function (Eq. \ref{eq:gamma}), we compute the 
constraint functions $\varphi_{\ell}(\mathbf{p})$ (Eqs \ref{eq:phi1}, \ref{eq:phi2},
\ref{eq:phi3}, \ref{eq:phi4}, \ref{eq:phi5}, \ref{eq:phi6} and \ref{eq:phi7}), 
$\ell = 1, \dots, 7$, by using the expressions which are written as sum of 
terms instead of those defined by sparse matrices.


\subsubsection{Setting the weights $\alpha_{1}-\alpha_{7}$}

Attributing values to the weights $ \alpha_{\ell} $ (Eq. \ref{eq:gamma}) is an important feature of our method. 
However, there is no analytical rule to define them and their values can be dependent on the particular characteristics of the type of geological setting where the method is being applied \cite[]{silva-2001}. 

At this point, we draw attention that the weights $ \alpha_{\ell}$ (Eq. \ref{eq:gamma}) are dimensional quantities. 
Note that the units of the data-misfit function (Eq. \ref{eq:misfit}) and the constraint functions 
(Eqs \ref{eq:phi1}, \ref{eq:phi2}, \ref{eq:phi3}, \ref{eq:phi4}, \ref{eq:phi5}, 
\ref{eq:phi6} and \ref{eq:phi7}), are nT$^{2}$ and $m^{2}$, respectively.
Because we set the unit of the goal function (Eq. \ref{eq:gamma}) as nT$^{2}$, the unit of the weights 
$ \alpha_{\ell} $ (Eq. \ref{eq:gamma}) is nT$^{2}$m$^{-2}$.

The physical dimensions of the weights $ \alpha_{\ell}$ cause the assignment of 
their values to be problem dependent. 
To make these weights comparable to each other, we normalize the $ \alpha_{\ell} $ 
values as follows:
\begin{equation}\label{eq:alphas}
\alpha_{\ell} = \tilde{\alpha}_\ell \frac{E_\phi}{E_\ell}, \quad \ell = 1,\dots, 7,
\end{equation}
where $\tilde{\alpha}_\ell$ is a positive scalar and $ E_\phi/E_\ell $ is a normalizing
factor allowing the $\tilde{\alpha}_\ell$ to be independent of the physical units used.
In Eq. \ref{eq:alphas}, $ E_\ell $ represents the 
trace of the Hessian matrix $\mathbf{H}_{\ell}$ (Eqs \ref{eq:phi1_hessian}, 
\ref{eq:phi2_hessian}, \ref{eq:phi3_hessian}, \ref{eq:phi4_hessian}, \ref{eq:phi5_hessian}, 
\ref{eq:phi6_hessian}, and \ref{eq:phi7_hessian}) of the $\ell$th constraining function 
$\varphi_{\ell}(\mathbf{p})$ (Eqs \ref{eq:phi1}, \ref{eq:phi2}, \ref{eq:phi3}, 
\ref{eq:phi4}, \ref{eq:phi5}, \ref{eq:phi6}, and \ref{eq:phi7}). 
The constant $E_\phi$ is the trace of the Hessian matrix 
$\mathbf{H}_{\phi}(\hat{\mathbf{p}}_{(0)})$ (Eq. \ref{eq:phi_hessian}) of the misfit function 
$\phi(\mathbf{p})$ (Eq. \ref{eq:misfit}) computed with the initial approximation $\hat{\mathbf{p}}_{(0)}$ 
for the parameter vector $ \mathbf{p} $ (Eq. \ref{eq:p-vector}) at the beginning of the inversion algorithm. 
Note that the trace of the Hessian matrix $\mathbf{H}_{\ell}$ is dimensionless and 
the trace of the Hessian matrix $\mathbf{H}_{\phi}(\hat{\mathbf{p}}_{(0)})$ has unit of nT$^{2}$m$^{-2}$.
Thus, the positive scalars $\tilde{\alpha}_\ell$ in Eq. \ref{eq:alphas} are dimensionless quantities.

According to this empirical strategy, the weights $ \alpha_{\ell} $ 
(Eq. \ref{eq:gamma}) are redefined using Eq. \ref{eq:alphas}, in which the weights
$\tilde{\alpha}_\ell$ are positive scalars which have no physical units and, 
consequently, are less dependent on the particular characteristics of the
interpretation geological setting.

%\subsubsection{Practical considerations}

The values attributed to the dimensionless weights $\tilde{\alpha}_{1} - \tilde{\alpha}_{7}$ 
(Eq. \ref{eq:alphas}) significantly impact the estimated models and cannot be 
automatically set without the interpreter’s judgment. 
Based on our practical experience, we suggest some 
empirical procedures for setting these parameters.

The parameters $\tilde{\alpha}_1$ and $\tilde{\alpha}_2$ impose prior information 
on the shape of the horizontal cross-section of the prisms. 
The first one forces all prisms to have a circular horizontal cross-section, while 
the second forces all prisms to have a similar horizontal cross-section.
Generally, their values vary from $10^{-5}$ to $10^{-3}$ and differ from
each other by one order of magnitude, at most.
The parameter $\tilde{\alpha}_3$ also varies from $10^{-5}$ to $10^{-3}$ and 
controls the relative position of adjacent prisms forming the model.
A high value privileges a vertical estimated body, whereas a small value 
tends to generate an inclined estimated body.

In comparison to $\tilde{\alpha}_1$, $\tilde{\alpha}_2$ and $\tilde{\alpha}_3$,
the other parameters usually have smaller values varying from $10^{-8}$ to $10^{-4}$.
The parameters $\tilde{\alpha}_4$ and $\tilde{\alpha}_5$ are used when a priori
information about the source is available at the study area.
The parameter $\tilde{\alpha}_6$ has a purely mathematical meaning and it is 
used only to obtain stable solutions for the inverse problem.
Its value is set to be as small as possible.
The parameter $\tilde{\alpha}_7$ controls the total-vertical extension of the 
the estimated body. 
The greater its value, the shallower the estimated depth to the bottom of the source will be 
and vice versa.
A general rule is starting with values 
$\tilde{\alpha}_1 = 10^{-4}$, $\tilde{\alpha}_2 = 10^{-4}$, $\tilde{\alpha}_3 = 10^{-4}$,
$\tilde{\alpha}_4 = 0$, $\tilde{\alpha}_5 = 0$, $\tilde{\alpha}_6 = 10^{-7}$, 
$\tilde{\alpha}_7 = 10^{-5}$ and change them to refine the results.
Finally, we stress that $\tilde{\alpha}_1 - \tilde{\alpha}_7$ (Eq. \ref{eq:alphas})
do not change along the iterations.

