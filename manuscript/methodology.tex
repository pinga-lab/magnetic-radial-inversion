\section{Methodology}\label{sec:metodo}

\subsection{Forward problem}

Let $\mathbf{d}^{o}$ be the observed data vector, whose $i$th element $d^{o}_{i}$, $i = 1, \dots, N$, is the total-field 
anomaly produced by a 3-D source (Fig. \ref{fig:obs}a) at the point $(x_{i}, y_{i}, z_{i})$ of a Cartesian coordinate 
system with $x$, $y$ and $z$ axes pointing to north, east and down, respectively. We assume that the direction of the 
total magnetization vector of the source is constant and known. 
We approximate the volume of the source by a set of $L$ vertically juxtaposed 3-D prisms 
(Fig. \ref{fig:obs}b) by following the same approach of \citet{oliveirajr-etal2011} and \citet{oliveirajr-barbosa2013}. 
The depth to the top of the shallowest prism is defined by $z_{0}$ and $m_{0}$ is the constant total-magnetization 
intensity of all prisms. 
The horizontal cross-section of each prism is described by a polygon with a fixed number 
$V$ of vertices equally spaced from $0^{\circ}$ to $360^{\circ}$, which are described in polar coordinates 
referred to an internal origin $O^{k}$ . 
The radii of the vertices ($r^{k}_{j}$, $j=1,\dots , V$, $k=1,\dots ,L$), the horizontal coordinates ($x_{0}^{k}$ and $y_{0}^{k}$, $k=1,\dots ,L$) 
of the origins $O^{k}$, $k=1,\dots ,L$, and the depth extent $dz$ of the $L$ vertically stacked prisms (Fig. \ref{fig:obs}b) are arranged in a 
$M \times 1$ parameter vector $\mathbf{p}$, $M = L (V + 2) + 1$, given by
%\begin{equation}
%\mathbf{p} = \left[ \begin{array}{@{}*{12}{c}@{}}
%r_{1}^{1} & \dots & r_{V}^{1} & x_{0}^{1} & y_{0}^{1} & \dots & r_{1}^{L} & \dots & r_{V}^{L} & x_{0}^{L} & y_{0}^{L} & dz \\ 
%\end{array} \right]^{\mathsf{T}} \: .
%\end{equation}
\begin{equation}
\mathbf{p} = \left[ \begin{array}{@{}*{12}{c}@{}}
{\mathbf{r}^{1}}^{\mathsf{T}} & x_{0}^{1} & y_{0}^{1} & \dots & {\mathbf{r}^{L}}^{\mathsf{T}} & x_{0}^{L} & y_{0}^{L} & dz \\
\end{array} \right]^{\mathsf{T}} \: ,
\label{eq:p-vector}
\end{equation}
where $^{\mathsf{T}}$ denotes transposition and $\mathbf{r}^{k}$ is a $V \times 1$ vector containing the radii $r^{k}_{j}$ 
of the $k$th prism.
Let $\mathbf{d} (\mathbf{p})$ be the predicted data vector, whose $i$th element 
\begin{equation}
d_{i} (\mathbf{p}) \equiv \sum\limits_{k=1}^{L} f_{i}^{k}(\mathbf{r}^{k}, x_{0}^{k}, y_{0}^{k}, dz, z_{1}^{k}, m_{0}), \quad i = 1, \dots, N \: ,
\label{eq:predicted-data-i}
\end{equation}
is the total-field anomaly produced by the ensemble of $L$ prisms at the $i$th observation point ($x_{i}, y_{i}, z_{i}$). 
In eq. \ref{eq:predicted-data-i}, $f_{i}^{k}(\mathbf{r}^{k}, x_{0}^{k}, y_{0}^{k}, dz, z_{1}^{k}, m_{0})$ is the total-field anomaly 
produced, at the observation point ($x_{i}, y_{i}, z_{i}$), by the $k$th prism, with depth to the top $z_{1}^{k} = z_{0} + (k-1)dz$.
We calculate $d_{i} (\mathbf{p})$ (eq. \ref{eq:predicted-data-i}) by using the Python package Fatiando a Terra \citep{uieda-etal2013}, 
which implements the formulas proposed by \cite{plouff1976}.

\subsection{Inverse problem}

Given a set of tentative values for depth to the top of the shallowest prism $z_{0}$ and for the intensity of the total-magnetization of the source
$m_{0}$, we solve a constrained non-linear problem to estimate the parameter vector $\mathbf{p}$ (eq. \ref{eq:p-vector}) by minimizing the 
objective function
\begin{equation}
\Gamma (\mathbf{p}) = \phi (\mathbf{p}) + \sum\limits^{7}_{\ell =1} \alpha_{\ell} \, \varphi_{\ell}(\mathbf{p}) \: ,
\label{eq:gamma}
\end{equation}
subject to
\begin{equation}
p_{l}^{min} < p_{l} < p_{l}^{max}, \quad l = 1, \dots, M \: ,
\label{eq:inequality-constraint}
\end{equation}
where $\varphi (\mathbf{p})$ is the data-misfit function given by
\begin{equation}\label{eq:misfit}
\phi (\mathbf{p}) = \frac{1}{N} \| \mathbf{d}^{o} - \mathbf{d}(\mathbf{p}) \|_{2}^{2} \: ,
\end{equation}
which represents the normalized squared Euclidean norm of the difference between the observed data vector $\mathbf{d}^{o}$ and 
the predicted data vector $\mathbf{d}(\mathbf{p})$, $\alpha_{\ell}$ is a small positive number representing the weight of the 
$\ell$th constraint function $\varphi_{\ell}(\mathbf{p})$ and $p_{l}^{min}$ and $p_{l}^{max}$ are, respectively, the lower and 
upper limits for the $l$th element $p_{l}$ of the parameter vector $\mathbf{p}$ (eq. \ref{eq:p-vector}). 
These limits are defined by the interpreter based on both the horizontal extent of the magnetic anomaly and the knowledge 
about the source.
We use the Levenberg–Marquardt method \citep[][ p. 240]{aster-etal2019} to minimize the objective function 
$\Gamma (\mathbf{p})$ (equation \ref{eq:gamma}) and introduce the inequality constraints 
(equation \ref{eq:inequality-constraint}) by using a strategy similar to that presented by \citet{barbosa-etal1999}.
The constraint functions $\varphi_{\ell}(\mathbf{p})$, $\ell = 1, \dots , 7$, used to obtain stable solutions and 
introduce a priori information about the source are defined below.

\begin{enumerate}
\item Smoothness constraint on the adjacent radii defining the horizontal section of each vertical prism. This constraint imposes that adjacent 
radii $r_{j}^{k}$ and $r_{j+1}^{k}$ within each prism must be close to each other. It forces the estimated prism to be approximately cylindrical. 
This constraint is given by
\begin{equation}
\varphi_{1}(\mathbf{p}) = \mathbf{p}^\mathsf{T}\mathbf{R}^\mathsf{T}_{1}\mathbf{R}_{1}\mathbf{p} \: ,
\label{eq:phi1}
\end{equation}
where 
\begin{equation}
\mathbf{R}_{1} = 
\begin{bmatrix}
\mathbf{S}_{1} & \mathbf{0}_{1} & \mathbf{0}_{1} & \cdots         & \mathbf{0}_{1} & \mathbf{0} \\
\mathbf{0}_{1} & \mathbf{S}_{1} & \mathbf{0}_{1} & \cdots         & \mathbf{0}_{1} & \mathbf{0} \\
\mathbf{0}_{1} & \mathbf{0}_{1} & \ddots         &                & \vdots         & \mathbf{0} \\
\vdots         & \vdots         &                & \ddots         & \mathbf{0}_{1} & \vdots     \\
\mathbf{0}_{1} & \mathbf{0}_{1} & \cdots         & \mathbf{0}_{1} & \mathbf{S}_{1} & \mathbf{0} \\
\end{bmatrix}_{LV \times M} \: ,
\label{eq:R1-matrix}
\end{equation}
where $\mathbf{0}$ is a $V \times 1$ vector with all elements equal to zero, $\mathbf{0}_{1}$ is a 
$V \times (V+2)$ matrix with all elements equal to zero and $\mathbf{S}_{1}$ is a matrix given by 
\begin{equation}
\mathbf{S}_{1} = 
\begin{bmatrix}
 1      & -1     &  0     & 0      & \cdots &  0     &  0     & 0      & 0      \\
 0      &  1     & -1     & 0      & \cdots &  0     &  0     & 0      & 0      \\
\vdots  & \vdots & \vdots & \vdots &        & \vdots & \vdots & \vdots & \vdots \\
 0      &  0     &  0     & 0      & \cdots & 1      & -1     & 0      & 0      \\
-1      &  0     &  0     & 0      & \cdots & 0      &  1     & 0      & 0      \\
\end{bmatrix}_{V \times (V+2)} \: .
\label{eq:S1-matrix}
\end{equation}


\item Smoothness constraint on the adjacent radii of the vertically adjacent prisms. This constraint imposes that adjacent radii 
$r_{j}^{k}$ and $r_{j}^{k+1}$ within vertically adjacent prisms must be close to each other. It forces the shape of all prisms to be 
similar to each other. 
This constraint is given by
\begin{equation}
\varphi_{2}(\mathbf{p}) = \mathbf{p}^\mathsf{T}\mathbf{R}^\mathsf{T}_{2}\mathbf{R}_{2}\mathbf{p} \: ,
\label{eq:phi2}
\end{equation}
where 
\begin{equation}
\mathbf{R}_{2} = 
\begin{bmatrix}
\mathbf{S}_{2} & -\mathbf{S}_{2} &  \mathbf{0}_{2} & \cdots         &  \mathbf{0}_{2} & \mathbf{0} \\
\mathbf{0}_{2} &  \mathbf{S}_{2} & -\mathbf{S}_{2} &                &  \mathbf{0}_{2} & \mathbf{0} \\
\vdots         &                 & \ddots          & \ddots         &                 & \vdots     \\
\mathbf{0}_{2} &  \mathbf{0}_{2} &                 & \mathbf{S}_{2} & -\mathbf{S}_{2} & \mathbf{0} \\
\end{bmatrix}_{LV \times M} \: ,
\label{eq:R2-matrix}
\end{equation}
where $\mathbf{0}$ is a $V \times 1$ vector with all elements equal to zero, $\mathbf{0}_{2}$ is a 
$V \times (V+2)$ matrix with all elements equal to zero and $\mathbf{S}_{2}$ is a matrix given by 
\begin{equation}
\mathbf{S}_{2} = 
\begin{bmatrix}
 1      & -1     &  0     & 0      & \cdots &  0     &  0     & 0      & 0      \\
 0      &  1     & -1     & 0      & \cdots &  0     &  0     & 0      & 0      \\
\vdots  & \vdots & \vdots & \vdots &        & \vdots & \vdots & \vdots & \vdots \\
 0      &  0     &  0     & 0      & \cdots & 1      & -1     & 0      & 0      \\
-1      &  0     &  0     & 0      & \cdots & 0      &  1     & 0      & 0      \\
\end{bmatrix}_{V \times (V+2)} \: .
\label{eq:S2-matrix}
\end{equation}

\item The source’s outcrop constraint. In the case of outcropping sources, this constraint imposes that the estimated horizontal cross-section of the shallowest prism must be close to the intersection of the geologic source with the known outcropping boundary. The matrix form of the this constraint is given by
\begin{equation}
\varphi_{3}(\mathbf{p}) = \left(\mathbf{A}\mathbf{p} - \mathbf{p}"\right)^\mathsf{T}\left(\mathbf{A}\mathbf{p} - \mathbf{p}"\right) ,
\end{equation}
where $\mathbf{p}"$ is a vector containing the parameters defining the polygon that represents the outcropping body given by
\begin{equation}
\mathbf{p}" = 
\begin{bmatrix}
r_{1}^{0} \\
r_{2}^{0} \\
\vdots \\
r_{M}^{0} \\
x_{0}^{0} \\
y_{0}^{0}
\end{bmatrix}_{(V+2)\times 1},
\end{equation}
and

\begin{equation}
\mathbf{A} = 
\begin{bmatrix}
\mathbf{I} & \hat{\mathbf{0}} \\
\end{bmatrix}_{(V+2)\times M},
\end{equation}
where $\mathbf{I}$ is an identity matrix with shape $V+2$ and $\hat{\mathbf{0}}$ is null matrix with shape $M -(V+2)$;

\item The source's horizontal location constraint. In the case of outcropping sources, this constraint imposes that the estimated horizontal Cartesian coordinates of the arbitrary origin within the shallowest prism must be as close as possible to the known horizontal Cartesian coordinates of a point on the outcropping body. The matrix form of the this constraint is given by
\begin{equation}
\varphi_{4}(\mathbf{p}) = \left(\mathbf{B}\mathbf{p} - \mathbf{p}'\right)^\mathsf{T}\left(\mathbf{B}\mathbf{p} - \mathbf{p}'\right) ,
\end{equation}
where
\begin{equation}
\mathbf{B} = 
\begin{bmatrix}
\mathbf{B}^{\sharp} & \hat{\mathbf{0}} \\
\end{bmatrix}_{2\times M} ,
\end{equation}

\begin{equation}
\textbf{B}^{\sharp} = 
\begin{bmatrix}
0 & \cdots & 0 & 1 & 0 \\
0 & \cdots & 0 & 0 & 1
\end{bmatrix}_{2\times (V+2)},
\end{equation}
and $\textbf{p}'$ is a vector containing the Cartesian coordinates of the horizontal location of the source given by

\begin{equation}
\textbf{p}' = 
\begin{bmatrix}
x^0_0 \\
y^0_0
\end{bmatrix}_{2\times 1} ;
\end{equation}

\item Smoothness constraint on the horizontal position of the arbitrary origins of the vertically adjacent prisms. This constraint imposes that the estimated horizontal Cartesian coordinates of vertically adjacent prisms must be close to each other. It forces the estimated prisms to be approximately vertically aligned. The matrix form of the this constraint is given by
\begin{equation}
\varphi_{5}(\textbf{p}) = \textbf{p}^\mathsf{T}\textbf{R}^\mathsf{T}_{5}\textbf{R}_{5}\textbf{p} ,
\end{equation}
where
\begin{equation}
\mathbf{R}_{5} = 
\begin{bmatrix}
\mathbf{R}^{-}_{5} & \mathbf{R}^{+}_{5} & \mathbf{0}^{\pm} & \mathbf{0}^{\pm} & \cdots & \mathbf{0}^{\pm} & \mathbf{0}^{\pm} & \mathbf{0}\\
\mathbf{0}^{\pm} & \mathbf{R}^{-}_{5} & \mathbf{R}^{+}_{5} & \mathbf{0}^{\pm} & \cdots & \mathbf{0}^{\pm} & \mathbf{0}^{\pm} & \mathbf{0}\\
\mathbf{0}^{\pm} & \mathbf{0}^{\pm} & \mathbf{0}^{\pm} &  &  & \vdots & \vdots & \vdots\\
\vdots & \vdots & \vdots &  &  & \vdots & \vdots & \vdots\\
\mathbf{0}^{\pm} & \mathbf{0}^{\pm} & \mathbf{0}^{\pm} & \cdots & \cdots & \mathbf{R}^{-}_{5} & \mathbf{R}^{+}_{5} & \mathbf{0}\\
\end{bmatrix}_{2(L-1)\times M},
\end{equation}
where $\mathbf{0}^{\pm}$ is a null matrix with the same shape of $\mathbf{R}^{-}_{5}$ and $\mathbf{R}^{+}_{5}$ which are given by
\begin{equation}
\textbf{R}^{-}_{5} = 
\begin{bmatrix}
0 & 0 & \cdots & 0 & -1 & 0 \\
0 & 0 & \cdots & 0 & 0 & -1 \\
\end{bmatrix}_{2\times (V+2)},
\end{equation}

\begin{equation}
\textbf{R}^{+}_{5} = 
\begin{bmatrix}
0 & 0 & \cdots & 0 & 1 & 0 \\
0 & 0 & \cdots & 0 & 0 & 1 \\
\end{bmatrix}_{2\times (V+2)};
\end{equation}

\item Minimum Euclidean norm constraint on the adjacent radii within each vertical prism. This constraint imposes that all estimated radii within each prism must be close to null values. The matrix form of the this constraint is given by
\begin{equation}
\varphi_{6}(\textbf{p}) = \textbf{p}^\mathsf{T}\textbf{C}^\mathsf{T}\textbf{C}\textbf{p},
\end{equation}
where
\begin{equation}
\textbf{C} = 
\begin{bmatrix}
\textbf{C}^{\sharp} & \mathbf{0} & \mathbf{0} & \cdots & \mathbf{0} & \mathbf{0}\\
\mathbf{0} & \textbf{C}^{\sharp} &  \mathbf{0} & \cdots & \mathbf{0} & \mathbf{0}\\
\mathbf{0} & \mathbf{0} & \ddots & & \vdots & \vdots\\
\vdots & \vdots & & \ddots & \vdots & \vdots\\
\mathbf{0} & \mathbf{0} &  \cdots & \cdots & \textbf{C}^{\sharp} & \mathbf{0}\\
\mathbf{0} & \mathbf{0} &  \cdots &  \cdots & \mathbf{0} & 0\\
\end{bmatrix}_{M\times M} ,
\end{equation}

\begin{equation}
\textbf{C}^{\sharp} = 
\begin{bmatrix}
1 & & & & \\
& \ddots &  & &  \\
&  & 1 & & \\
&  & & 0 &  \\
&  & & & 0 \\
\end{bmatrix}_{(V+2)\times (V+2)};
\end{equation}

\item Minimum Euclidean norm constraint on the depth extent of all prisms. This constraint imposes that the estimated depth extent of the prisms must be close to a null value. The matrix form of the this constraint is given by
\begin{equation}\label{eq:phi7}
\varphi_7 = \mathbf{p}^\mathsf{T}\mathbf{D}^\mathsf{T}\mathbf{D}\mathbf{p},
\end{equation}
where
\begin{equation}
\mathbf{D} =
\begin{bmatrix}
 0 & \cdots  & 0 \\
 \vdots & \ddots & \vdots\\
 0  & \cdots  & 1\\
\end{bmatrix}_{M\times M}.
\end{equation}

\end{enumerate}

Most of these constraints are defined by using the Tikhonov regularizing functions of order zero and one \citep{aster-etal2019}, by following the same approach presented by \citet{oliveirajr-etal2011}. Here, we present the constraint on the depth extent of all prisms.