\section{Conclusions}

We have developed a 3-D radial inversion of total-field anomaly data to estimate the shape of an isolated 3-D geological body assuming the knowledge about its total-magnetization direction. We approximate the 3-D body by a set of vertically stacked right polygonal prisms with fixed depth to the top, thickness and total-magnetization vector. The horizontal cross-section of each prism is a polygon described by a given number of equally spaced radii from 0$^\circ$ to 360$^\circ$ connecting an arbitrary origin to the vertices of the polygon. We estimate the depth to the top and the total-magnetization intensity of the source using a validation test based on the minimization of the goal function. In this test, we run an inversion for a combination of ranges of depths to the top and total-magnetization intensities. The lowest value of the goal function indicates the optimal depth to the top and total-magnetic intensity. Our method defines the geometry of the 3-D geological source by estimating via an iterative nonlinear inversion algorithm the Cartesian coordinates of the arbitrary origin, the radii and the thickness of the prisms. For all the inversions, we set a cylinder shape as the initial approximate that involves part of the horizontal shape of the anomaly. To stabilize the inverse problem we introduced a set of seven constraints on the source shape.

This method is an extension of previous work that applied the set of polygonal prisms to estimate the shape of 3-D geological source by inverting gravity and gravity-gradient data. However, both of them assume th knowledge of the depth to the top and the physical property of the source. Also, they proposed a method to estimate by successive inversions the depth to the bottom of the source. In this work, we proposed a validation test to estimate both depth to the top and the physical property. Moreover, our method estimates the depth to bottom of the interpretation model introducing the thickness of the prisms as a parameter in the inverse problem. Unfortunately, our method does not guarantee the uniqueness of the solution due to the ambiguity of potential fields problems.

We have applied our method to an synthetic test simulating a symmetric source retrieving accurately the true model geometry. This example illustrates the efficiency of the constraints on the source's shape in an ideal case. Moreover, we have simulated a complex body to evaluate the performance of our method in a more realistic case. In this example, our method was capable to estimate a complex geometry with dipping and a variable shape along the depth. Both test were successfully achieved obtaining an estimated model that retrieve the source's shape and fits the data.

We applied our method to the total-field anomaly data of the alkaline-carbonatitic complex of Anitapolis, Santa Catarian state, Brazil. The estimated model shows a very complex geometry for the magnetic source that fits the observed data. Based on the map and histogram of residual, we interpret that this estimated model can be a possible geometry for the geological body.

A possible extension of this work is the inversion of a very elongated magnetic source or multiple bodies. In addition, an inversion method combining gradient-based and heuristic approaches could be applied to estimate the optimal regularization weights for the set of seven constraints, overcoming problems with local minima.