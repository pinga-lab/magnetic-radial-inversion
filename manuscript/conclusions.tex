\section{Conclusions}

We have developed a total-field anomaly nonlinear inversion to estimate the shape 
of an isolated 3-D geological body assuming the knowledge about its 
total-magnetization direction. We approximate the body by a set of vertically 
stacked right prisms. The horizontal cross-section of each prism is a polygon defined by a given number of
equi-angularly spaced vertices from $0^\circ$ to $360^\circ$. 
We peform our inversion for a set of tentative values for the depth to the top 
of the shallowest prism and the total-magnetization intensity of all prisms. 
For each tentative pair of depth to the top and total-magnetization intensity,
our method estimates the geometries of the cross-sections (the radii associated with the polygon vertices), the thickness and the horizontal positions of the prisms. 
The estimated bodies producing the smallest values of the goal function form the 
set of candidate solutions.
Our method is an extension of previous works developed for retrieving the geometry 
of 3-D bodies by inverting gravity and gravity-gradient data. 
We not only adapted the previous methods for interpreting total-field anomaly data,
but also generalized them to include the depth to the top and depth extension of 
the prisms forming the interpretation model.

Results obtained with synthetic data produced by a simple symmetric source 
and by a realistic geological source, in the presence of regional field, 
with variable dip and shape with depth, show that our method is able to retrieve 
the shape of the source and fit the data in both cases.
We applied our method to interpret a total-field anomaly data over the 
alkaline-carbonatitic complex of Anit{\'a}polis, in southern Brazil. 
We obtained two candidate models having similar shapes, depths to the top and 
total-magnetic intensities, all of them consistent with the available 
geological information. 
Both estimated models suggest that the emplacement of the 
Anit{\'a}polis complex seems to be controlled by a nearly N30W-trending fault
in agreement with previous studies.
It is important to bear in mind, however, the possible bias in the geometry of the
estimated body due to errors in the total-magnetization direction used as 
a priori information.

Finally, we stress that our method is not strictly to invert the total-field anomaly; 
rather, it can be adapted to invert single and multiple magnetic field components. 
Possible extensions of this work is the inversion of elongated and/or multiple 
sources. 
In addition, the combination of gradient-based and heuristic optimization 
methods could be applied to estimate optimal regularization weights and 
overcome problems with local minima.