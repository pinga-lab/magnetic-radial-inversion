\section{Conclusions}

We have developed a total-field anomaly nonlinear inversion to estimate the shape 
of an isolated 3-D geological body assuming the knowledge about its 
total-magnetization direction. We approximate the body by a set of vertically 
stacked right prisms. The horizontal cross-section of each prism is a polygon 
defined by a given number of equally spaced vertices from 0$^\circ$ to 360$^\circ$. 
We peformed our inversion for a set of given depths to the top and total-magnetization 
intensities these results illustrate the inherent ambiguity of potential-field methods in 
retrieving both the physical-property distribution and volume of the sources. 
In this case, some a priori information must 
be used to constraint the range of reliable solutions. For each depth to the top and total-magnetization intensity,
our method estimates the geometry of the cross-sections (the radii associated with the polygon vertices), the thickness and 
the horizontal position of the prisms. The estimated model approximates the 3-D geological body by 
solving a constrained nonlinear magnetic inversion.
The estimated bodies producing the smallest values of the goal function form the 
set of candidate solutions that yields an acceptable data fitting satisfying stabilizing functions. 
To obtain stable solutions, we introduce a set of seven constraints on the source 
shape.
Our method is an extension of previous works developed for retrieving the geometry 
of 3-D bodies by inverting gravity and gravity-gradient data. 
We not only adapted the previous methods for interpreting total-field anomaly data,
but also generalize them to include the depth to the top and depth extension of 
the prisms among the estimated parameters.

Applications to synthetic data produced by a simple symmetric source 
illustrate the efficiency of our method in an ideal case. Moreover, the results 
obtained with synthetic data produced by a complex source, with variable dip and 
shape along depth, show that our method can also be used to interpret magnetic 
data produced by a realistic geological source. 
Both tests with synthetic data show that our method is able to retrieve the 
source's shape and fit the data.

We applied our method to interpret a total-field anomaly data over the 
alkaline-carbonatitic complex of Anit{\'a}polis, in southern Brazil. 
We obtained two candidate models having similar shapes, depths to the top and 
total-magnetic intensities, all of them in close agreement with the available 
a priori information. Both estimated models suggest that the emplacement of the 
Anit{\'a}polis complex seems to be controlled by a nearly N30W-trending fault at depth.
This is in agreement with previous studies at the study area.
Possible extensions of this work is the inversion of elongated and/or multiple 
sources. In addition, an the combination of gradient-based and heuristic optimization 
methods could be applied to estimate optimal regularization weights, 
overcoming problems with local minima.