\section{Application to field data}

We applied our method to interpret the total-field anomaly data provided by CPRM and acquired by Lasa Prospecções S.A. over the Anitápolis complex in the state of Santa Catarina, Brazil (colocar mapa). The total-field anomaly data were corrected from daytime variation and subtracted from the main field of the Earth using the IGRF. According to Hartmann (citar), Anitápolis is an intrusive complex formed by alkaline-carbonatitic rocks identified using gammaspectrometry and aeromagnetometry of the Brazilian shield. The flight was at an elevation of 100 m above the terrain, the N–S and E–W lines were spaced 500 m and 10,000 m, respectively. We processed the data by applying a regional separation using a second-order polynomial fit. To attenuate the non-dipolar effects present in the data, we applied the equivalent layer to continue the anomaly upward to a constant height $z=-2000$ m (Fig. \ref{fig:real_data}a). The upward continued data were calculated in a regular grid of $50\times50$ points equally spaced from 6916 km to 6926 km in $x$ axis and 683 km to 693. in $y$ axis. The main field direction in the area at that time has an inclination and declination $(-37.05^\circ, -18.17^\circ)$.

To define the total-magnetization direction of the interpretation model, we used the reduction to the pole (RTP) technique with the main field direction in the area. The RTP result (not shown) indicates that the magnetic source has purely induced magnetization. The interpretation model is formed by an ensemble of $L = 8$ prisms, each one with number of polygon vertices $V = 30$ ($k = 1, \dots , 10$) describing the horizontal cross-sections of the polygons. We set the origin of the initial approximation $(x_0^k, y_0^k) = (6921, 688)$ km and the radii $r_j^k = 1500$ m so that it involves the center and great part of the positive and negative pols of the anomaly. The depth extent of the prisms is $dz = 700$ m that returns an total-field anomaly in the same range of the observed data. Following the approach for the synthetic applications (sec. \ref{sec:synthetic}), we have inverted the real total-field anomaly (Fig. \ref{fig:real_data}a) obtaining 36 different models. Each model was obtained by using a different pair of depth to the top $z_0$ and total-magnetization intensity $m_0$ (Fig. \ref{fig:real_map}). For all the 36 models, the initial approximate has the same cylindrical shape and the regularization weights $\tilde{\alpha}_1 = 10^{-3}$, $\tilde{\alpha}_2 = 10^{-4}$, $\tilde{\alpha}_3 = 10^{-3}$, $\tilde{\alpha}_4 = 0$, $\tilde{\alpha}_5 = 0$, $\tilde{\alpha}_6 = 10^{-6}$, and $\tilde{\alpha}_7 = 10^{-5}$. Figs \ref{fig:simple_map} and \ref{fig:complex_map} show that lowest values for $\Gamma(\mathbf{p})$ (eq. \ref{eq:gamma}) indicate the pair of $m_0$ and $z_0$ is close to the true one. Therefore, we can interpret that the estimated model that has the pair $z_0 = -900$ m and $m_0 = 5$ A/m (red triangle in Fig. \ref{fig:real_map}) is more realistic than the others. This estimated model produces the best data fit (Fig. \ref{fig:real_result}a) among the 36 inversions. The acceptable fit is confirmed by the histogram of residuals in the inset of the \ref{fig:real_result}a with $ \mu $ close to zero and a low standard deviation similar to the synthetic applications. Fig. \ref{fig:real_result}b shows the cylinder used as the initial approximate for the inversion. Figs \ref{fig:real_result}c and d show the estimated model (red prisms) for this application. Our method estimated a body northwest-southeast elongated.