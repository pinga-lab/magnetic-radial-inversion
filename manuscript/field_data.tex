\section{Application to field data}

We have applied our method to interpret airborne magnetic data provided by 
Geological Survey of Brazil (CPRM) over the Anit{\'a}polis complex, 
in southern Brazil. 
The airborne survey was flown with north-south and east-west lines spaced by $500$ m and $10\,000$ m from each other, respectively. 
The total-field anomaly data (Fig. \ref{fig:real_data}a) were corrected from diurnal correction and from the main geomagnetic field using the IGRF. 
The inclination, declination and intensity of the main geomanetic field at the study area, 
for the period of the survey, are $-37.05^{\circ}$, $-18.17^{\circ}$ and 
$\approx 22 \, 768 $ nT, respectively.
Fig. \ref{fig:real_data}b shows a regional field obtained by fitting, in the least-squares sense, a first-order polynomial to the original data (Fig. \ref{fig:real_data}a).
Figs \ref{fig:real_data}c and \ref{fig:real_data}d show the geometric height
We have subtracted 800 m from their values and after the invertion the estimated model is corrected.  

After removing a first-order polynomial (Fig. \ref{fig:real_data}b) from the original data  
(Fig. \ref{fig:real_data}a) using a least-squares polynomial fitting, we obtain the residual total-field anomaly shown in Fig. \ref{fig:anitapolis_rtp}a.
The Anit{\'a}polis alkaline-carbonatitic complex forms a circular concentric body 
($\approx 6$ km$^{2}$ in area) containing magnetite as part of its mineralogical 
composition. 
It intruded into the Late Proterozoic leucogranites of the Dom Feliciano 
mobile belt in the Early Cretaceous ($132$ Ma), apparently concomitant with the 
voluminous flood tholeiitic basalts of the Serra Geral Formation ($133$-$130$ Ma) 
at the southern side of the Paran{\'a} Basin \citep{gibson-1999, scheibe-etal2005}.
As pointed out by \citet{GOMES2018}, there is still some debate about the emplacement 
of the Anit{\'a}polis alkaline-carbonatitic complex. 
\citet{melcher-coutinho1966} pointed out the influence of N-S-trending faults.
\citet{horbach-marimon1980} affirmed that the Anit{\'a}polis complex is controlled by 
a large N30W lineament. 
\citet{scheibe-etal2005} considered that it is roughly emplaced along the E-W Rio 
Uruguay Lineament. 
According to \citet{riccomini-etal2005}, the Anit{\'a}polis 
complex does not show a clear structural control. 

By using the methodology of \cite{reis-etal2020}, we set the total-magnetization direction of the interpretation model with inclination $ I=-21^\circ $ and declination $ D=-11^\circ $. These values were estimated by \cite{reis-seg-2019}. 
We have accepted this estimated magnetization direction because the calculated RTP anomaly 
(Fig. \ref{fig:anitapolis_rtp}b) of the residual total-field anomaly 
(Fig. \ref{fig:anitapolis_rtp}a) is predominantly positive and decays to zero toward
the borders of the study area \citep{reis-etal2020, reis-seg-2019}.

This total-magnetization direction indicates the presence of remanent magnetization.
Laboratory measurements on rock samples obtained at the Jacupiranga complex,
another alkaline complex located northward of the study area, 
with the same age as the Anit{\'a}polis complex, 
show total-magnetization intensities values varying from approximately 
$0.01$ to $29.90$ A/m \citep[][ tb. 1]{valdivia-2009}.
We used these values as a priori information to set the range of possible values 
for the total-magnetization intensity $m_{0}$ in the Anit{\'a}polis complex.

We used an interpretation model formed by $L = 6$ prisms, each one with 
$V = 20$ vertices defining their horizontal cross-sections.
We inverted the residual total-field anomaly (Fig. \ref{fig:anitapolis_rtp}a) for each 
pair of $m_0$ and $z_0$ shown in Fig. \ref{fig:anitapolis_rtp}c, resulting in $100$ estimated 
models. 
For all models, we set the same initial approximation $\hat{\mathbf{p}}_{(0)}$ 
(red prisms in Figs \ref{fig:real_result2}b and \ref{fig:real_result}b) 
with origin at $(x_0^k, y_0^k) = (6\,921, 688)$ km, constant radii $r_j^k = 700$ m for 
all vertices forming all prisms and the same constant thickness $dz = 900$ m.
We also set the same weights which were used in the synthetic tests, i.e., 
$\tilde{\alpha}_1 = 10^{-4}$, 
$\tilde{\alpha}_2 = 10^{-3}$, 
$\tilde{\alpha}_3 = 10^{-4}$, 
$\tilde{\alpha}_4 = 0$, 
$\tilde{\alpha}_5 = 0$, 
$\tilde{\alpha}_6 = 10^{-8}$, and 
$\tilde{\alpha}_7 = 10^{-5}$ (eq. \ref{eq:alphas}). 
Fig. \ref{fig:anitapolis_rtp}c shows that there is a region (blue region) containing 
candidate solutions producing small values of the goal function 
$\Gamma(\mathbf{p})$ (eq. \ref{eq:gamma}), with different values of $m_0$ and $z_0$.
Two estimates produced the smallest values for $ \Gamma(\mathbf{p}) $ (eq. \ref{eq:gamma}). 
The magenta diamond in Fig. \ref{fig:anitapolis_rtp}c represents the 
non-outcropping estimated model shown in Fig. \ref{fig:real_result2}. 
This model produces a reasonable data fit (Fig. \ref{fig:real_result2}a). 
It has a volume $ 6.94 $ km$ ^3 $, total thickness of $ 3\,045.69 $ m 
($ dz = 510.95 $), depth-to-the-top $z_0 = 20$ m and total-magnetization intensity 
$m_0 = 16.0$ A/m.
The white diamond in Fig. \ref{fig:anitapolis_rtp}c represents the outcropping estimated  
model shown in Fig. \ref{fig:real_result}. This model is similar to that shown in 
Fig. \ref{fig:real_result2}. It has a volume $ 7.24 $ km$ ^3 $, total thickness 
$ 3\,032.65 $ m ($ dz = 505.44 $), depth-to-the-top $z_0 = 0$ m and 
total-magnetization intensity $m_0 = 15$ A/m.  
In comparison with the non-outcropping model shown in Fig. \ref{fig:real_result2}, the 
alternative model shown in Fig. \ref{fig:real_result} has very similar geometry,
but it outcrops.

Finally, we stress that both total-magnetization intensities $m_0 = 16$  A/m and  $m_0 = 15$ A/m 
are within the range found by  \citet{valdivia-2009} for intrusions located at the Jacupiranga complex.
Both estimated models show a nearly N30W elongated body with high dip 
along depth (Figs \ref{fig:real_result2} and \ref{fig:real_result}), 
which coincides with the topographic low observed in Fig \ref{fig:real_data}c. 
Hence, these estimates agree with the interpretation proposed by 
\citet{horbach-marimon1980}, that considered the presence of N30W-trending fault 
controlling the Anit{\'a}polis complex.