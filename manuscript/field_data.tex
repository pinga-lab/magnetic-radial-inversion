\section{Application to field data}

We applied our method to interpret the total-field anomaly data provided by CPRM and acquired by Lasa Prospecções S.A. over the Anitápolis complex in the state of Santa Catarina, Brazil (colocar mapa). The total-field anomaly data were corrected from daytime variation and subtracted from the main field of the Earth using the IGRF. Anitápolis is an intrusive complex in southern Brazil formed by alkaline-carbonatitic rocks \cite[]{GOMES2018}. The flight was at an elevation of 100 m above the terrain, the N–S and E–W lines were spaced 500 m and 10,000 m, respectively. Fig \ref{fig:real_data}b shows GPS height of the acquisition and the horizontal coordinates referred to the WGS84 datum. We processed the data by applying a regional separation using a second-order polynomial fit. To attenuate the non-dipolar effects present in the data, we applied the equivalent layer \cite[]{dampney1969,emilia1973} to continue the anomaly upward to a constant height $z=-2000$ m (Fig. \ref{fig:real_data}a). The upward continued data were calculated in a regular grid of $50\times50$ points equally spaced from 6916 km to 6926 km in $x$ axis and 683 km to 693. in $y$ axis. The main field direction in the area at that time has an inclination and declination $(-37.05^\circ, -18.17^\circ)$.

To define the total-magnetization of the interpretation model, we used the reduction to the pole technique (RTP) with the main field direction in the area. The RTP result (not shown) indicates that the magnetic source has purely induced magnetization. \cite{valdivia-2009} measured the magnetic susceptibility $\chi$ of carbonatitic samples of Jacupiranga alkaline complex, another southern Brazil complex in São Paulo state. They have found that the magnetic susceptibility for those samples varies from $\chi = 1.06\times 10^{-3}$ SI to $\chi=161.05\times 10^{-3}$ SI \cite[][ tb. 1]{valdivia-2009}. There are carbonatites with similar ages in both Jacupiranga and Anitápolis complexes \cite[]{gibson-1999}. So, we based the values for $m_0$ using this reference and the main magnetic field intensity of the Earth in the area $ \approx 22768 $ nT. The magnetic susceptibility ($ \chi $) range used in this application is the upper axis in Fig. (\ref{fig:real_map}) which is equivalent to the $ m_0 $ axis.

The interpretation model is formed by an ensemble of $L = 8$ prisms, each one with number of polygon vertices $V = 30$ ($k = 1, \dots , 10$) describing the horizontal cross-sections of the polygons. We set the origin of the initial approximation $(x_0^k, y_0^k) = (6921, 688)$ km and the radii $r_j^k = 1500$ m so it involves the center and great part of the positive and negative pols of the anomaly. The depth extent of the prisms is $dz = 700$ m that returns an total-field anomaly in the same range of the observed data. Following the approach for the synthetic applications (sec. \ref{sec:synthetic}), we have inverted the real total-field anomaly (Fig. \ref{fig:real_data}a) obtaining 100 different models. Each model was obtained by using a different pair of depth to the top $z_0$ and total-magnetization intensity $m_0$ (Fig. \ref{fig:real_map}). The range of $m_0$ and $z_0$ in Fig. \ref{fig:real_map} were defined empirically, however, the $m_0$ range was based on the typical values of magnetic susceptibility for alkaline-carbonatitic rocks and the main field intensity for the study area. For all the 100 models, the initial approximate has the same cylindrical shape and the regularization weights $\tilde{\alpha}_1 = 10^{-3}$, $\tilde{\alpha}_2 = 10^{-4}$, $\tilde{\alpha}_3 = 10^{-3}$, $\tilde{\alpha}_4 = 0$, $\tilde{\alpha}_5 = 0$, $\tilde{\alpha}_6 = 10^{-6}$, and $\tilde{\alpha}_7 = 10^{-5}$. Figs \ref{fig:simple_map} and \ref{fig:complex_map} show that lowest values for $\Gamma(\mathbf{p})$ (eq. \ref{eq:gamma}) indicate the pair of $m_0$ and $z_0$ is close to the true one.

Therefore, we can interpret that the estimated model that has the pair $z_0 = -1200$ m and $m_0 = 2.5$ A/m (cyan diamond in Fig. \ref{fig:real_map}) is more compatible with the possible values of magnetic susceptibility $ \chi $ (upper axis in Fig. \ref{fig:real_map}). We do not have evidences of an outcropping source for this anomaly, but there are evidences of outcropping intrusions in the area of the Anitápolis complex \cite[]{gibson-1999}. The depth to the top $z_0 = -1200$ indicates a very shallow source or a possible outcropping source. This estimated model produces a very low value for $ \Gamma(\mathbf{p}) $ (\ref{eq:gamma}) whose data fit is shown in Fig. \ref{fig:real_result}a. The acceptable data fit is confirmed by the histogram of residuals in the inset of the \ref{fig:real_result}a with $ \mu $ close to zero and a low standard deviation similar to the synthetic applications. Fig. \ref{fig:real_result}b shows the cylinder used as the initial approximate for the inversion. Figs \ref{fig:real_result}c and d show the estimated model (red prisms) for this application. Our method estimated a body northwest-southeast elongated with volume $ 52.89 $ km$ ^3 $ and total depth extent $ 5434.35 $ m ($ dz=679.29 $). In addition, the magenta diamond in Fig. \ref{fig:real_map} is the estimated model that produces the lowest value for $ \Gamma(\mathbf{p}) $ (\ref{eq:gamma}). Nevertheless, both data fit and the estimated model (Figs \ref{fig:real_result2}a, c, and d) are very close to the cyan diamond solution (Fig. \ref{fig:real_result}) in Fig. (\ref{fig:real_map}). The volume and total depth extent for this solution (Figs \ref{fig:real_result2}c and d) are, respectively, $ 48.37 $ km$ ^3 $ and $ 5709.20 $ m ($ dz=713.65 $). Both estimated models are very similar to each other, but the first one thicker and shallower than the second one. This difference is given by the inversion in order to compensate the estimated magnetic momentum for different values of $ z_0 $ and $ m_0 $. So, considering the data fit and its histogram both solutions can be a possible geometry of this magnetic source in the Anitápolis complex. To confirm this estimated model, more prior geological information about this geological anomalous source must be included.