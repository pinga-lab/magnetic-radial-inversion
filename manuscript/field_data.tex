\section{Application to field data}

There is a region in the central part of Brazil where are occurrences of Cretaceous alkaline rocks along a lineament NW-SE. Authors have been studying this region since the 60s classifying it with different nomenclatures throughout the years. \citet{junqueirabrod-etal2002} have proposed the nomenclature Goiás alkaline province (GAP) for the alkaline rocks close to Rio Verde and Iporá cities (in Goiás state). Mafic to ultramafic alkaline rocks form GAP complex surrounded by a Precrambian basement and Phanerozoic sedimentary rocks from Paraná basin. An aeromagnetic survey was flown in 2004 with an almost constant height of $100$ m from the terrain covering the region.  The N-S lines are spaced from 500 m and the tie E-W tie-lines are spaced from $5000$ m. The flight height was approximately constant at $100$ m above the terrain, and the interval between the measurements was $0.1$ s, this interval resulted in one measurement at each approximately $8$ m. However, we decimated the data by removing measurements that the interval between them is 88 m (Figure \ref{fig:real_data}). The diurnal variation correction and the Earth's main field subtraction were made on the data. The Earth's main field was subtracted using the International Geomagnetic Reference Field (IGRF) evaluated at the 2004.62 epoch, with declination $-18.5$º and inclination $-19.5$º.

\citet{oliveirajr-etal2015} estimated the source's total magnetization direction in the alkaline complex of Diorama under the premise of approximated spherical sources. They validated the estimated magnetization direction (inclination $-71.4$º and declination $-23.4$º) by computing the reduction to the pole of magnetic anomalies over the alkaline complexes of Diorama and Montes Claros de Goiás. To attenuate the non-dipolar effects present in the data, we applied the equivalent layer \citep{dampney1969,emilia1973} to continue the anomaly upward to a constant normal height of 1100m. We applied our method on the Diorama upward continued total-field anomaly (not shown) by using an initial guess (blue prisms in Figure \ref{fig:real_result}) with $L=10$ prisms, each one with $M=30$ vertices, depth to the top $z_0=50$ m, $dz = 800$ m, and all radii are equal to $100$ m, total magnetization intensity $15$ A/m and direction equal to that estimated by \citet{oliveirajr-etal2015}. The estimated source (Figure \ref{fig:real_result}) reaches a maximum depth of about 6 km and has a very complex geometry. Although the data fitting produced by the estimated source may seem unacceptable because the large residuals (Figure \ref{fig:real_result}d) which range from $-64$ to $64$ nT, we stress that the misfit value is about to $9\%$.