\section{Application to field data}

We have applied our method to interpret airborne magnetic data provided by 
Geological Survey of Brazil (CPRM) over the Anitápolis complex, in the Santa Catarina
state, Brazil. The total-field anomaly data were corrected from daytime variation and 
subtracted from the main magnetic field using the IGRF. 
The inclination, declination and intensity of the main field at the study area, 
for the period of the survey, are $-37.05^{\circ}$, $-18.17^{\circ}$ and 
$\approx 22 \, 768 $ nT, respectively.
Anitápolis is an intrusive complex in southern Brazil formed by alkaline-carbonatitic 
rocks \cite[]{GOMES2018}. The flight was at an elevation of 100 m above the terrain, 
with N-S and E-W lines spaced by $500$ m and $10,000$ m from each other, respectively. 
Fig. \ref{fig:real_data}b shows the geometric height (referred to the WGS84 ellipsoid) 
and UTM horizontal coordinates of the data.

To isolate the target total-field anomaly, we have applied 
a regional separation using a second-order polynomial fit. We have also continued 
the anomaly upward to a constant height $z=-2000$ m (Fig. \ref{fig:real_data}a) by 
using the equivalent-layer technique \cite[]{dampney1969,emilia1973,oliveirajr-barbosa2013}. 
The upward-continued data were calculated on a regular grid of $50 \times 50$ points 
equally spaced from $6916$ to $6926$ km along the $x$ axis and from 
$683$ to $693$ km along the $y$ axis. 
To define the total-magnetization direction of the interpretation model, 
we applied the method proposed by \citet{oliveirajr-etal2015} and verified the 
result using the reduction to the pole technique (RTP). The results (not shown) 
indicate that the total magnetization of the source has the same direction of the 
main field in the study area, suggesting a purely induced magnetization.

Previous works \citep[e.g., ][]{gibson-1999} found carbonatites with similar ages in the 
Anitápolis and Jacupiranga complexes, where the last one is another alkaline complex 
located at southern Brazil, in the São Paulo state.
Measurements of the magnetic susceptibility $\chi$ made at the Jacupiranga complex 
show values varying from $\chi = 1.06 \times 10^{-3}$ SI to 
$\chi = 161.05 \times 10^{-3}$ SI \citep[][ tb. 1]{valdivia-2009}.
Hence, we used these values and the main magnetic field intensity in the study area 
($\approx 22 \, 768 $ nT) to constraint the total-magnetization intensity $m_{0}$ 
values used by our method.
The magnetic susceptibility ($ \chi $) and the corresponding total-magnetization 
intensity $m_{0}$ ranges used in our application are shown in Fig. \ref{fig:real_map}.

We used an interpretation model formed by $L = 8$ prisms, each one with 
$V = 30$ ($k = 1, \dots , 10$) vertices defining their horizontal cross-sections.
We inverted the observed total-field anomaly (Fig. \ref{fig:real_data}a) for each 
pair of $m_0$ and $z_0$ shown in Fig. \ref{fig:real_map}, resulting in $100$ estimated 
models. 
For all models, we set an initial approximation $\hat{\mathbf{p}}_{(0)}$ with origin 
at $(x_0^k, y_0^k) = (6921, 688)$ km, constant radii $r_j^k = 1500$ m for 
all vertices forming all prisms and the same constant depth extent $dz = 700$ m.
We also used the same weights $\tilde{\alpha}_1 = 10^{-3}$, $\tilde{\alpha}_2 = 10^{-4}$, 
$\tilde{\alpha}_3 = 10^{-3}$, $\tilde{\alpha}_4 = 0$, $\tilde{\alpha}_5 = 0$, 
$\tilde{\alpha}_6 = 10^{-6}$, and $\tilde{\alpha}_7 = 10^{-5}$ (eq. \ref{eq:alphas}). 

Fig. \ref{fig:real_map} shows that, similarly to the results obtained with the 
complex model (Fig. \ref{fig:complex_map}), there is an elongated region containing 
candidate solutions producing small values for the goal function 
$\Gamma(\mathbf{p})$ (eq. \ref{eq:gamma}), with different values of $m_0$ and $z_0$.
The magenta diamond in Fig. \ref{fig:real_map} represents the estimated model 
shown in Fig. \ref{fig:real_result2}. 
This model produces the lowest value for $ \Gamma(\mathbf{p}) $ (\ref{eq:gamma}),
it has a volume $ 48.37 $ km$ ^3 $, total depth extent of $ 5709.20 $ m 
($ dz = 713.65 $), depth-to-the-top $z_0 = -1150$ m and total-magnetization intensity 
$m_0 = 3.0$ A/m. This value corresponds to a magnetic susceptibility 
$chi = 0.16$ SI.
Fig. \ref{fig:real_map}a shows that this model produces a very good data fit.
This estimated depth-to-the-top $z_0$ indicates a non-outcropping source, 
which is compatible with a priori information a the study area. 
We do not have evidences of an outcropping source for this anomaly, 
although there are outcropping intrusions in the area of the Anitápolis 
complex \cite[]{gibson-1999}.
The estimated magnetic susceptibility $ \chi $ is also compatible with the 
available a priori information. It is close to the upper limit found by 
\citet{valdivia-2009} in the Jacupiranga complex.

The cyan diamond in Fig. \ref{fig:real_map} represents the alternative model 
show in Fig. \ref{fig:real_result}. This model is similar to that shown in 
Fig. \ref{fig:real_result2}. It has a volume $ 52.89 $ km$ ^3 $, total depth extent 
$ 5434.35 $ m ($ dz = 679.29 $), depth-to-the-top $z_0 = -1200$ m and 
total-magnetization intensity $m_0 = 2.5$ A/m, which corresponds to a magnetic 
susceptibility $chi = 0.13$ SI. 
In comparison to the model shown in Fig. \ref{fig:real_result2}, the alternative model 
shown in Fig. \ref{fig:real_result} has a shallower (but still non-outcropping) top
and a smaller estimated magnetic susceptibility. As a consequence, the alternative 
model has a greater volume in order to produce an equally good data fit 
(Fig. \ref{fig:real_result}a).

Both estimated models (Figs \ref{fig:real_result2} and \ref{fig:real_result}) show 
a northwest-southeast elongated body with variable dip along depth.
They have estimated magnetic susceptibility and depth-to-the-top compatible with 
available a priori information and produce very good data fits.
Hence, both models represent the possible geometry of the Anitápolis complex. 