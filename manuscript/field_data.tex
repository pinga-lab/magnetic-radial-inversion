\section{Application to field data}

We have applied our method to interpret airborne magnetic data provided by 
Geological Survey of Brazil (CPRM) over the Anit{\'a}polis complex, 
in southern Brazil. 
The airborne survey was flown with north-south and east-west lines spaced by $500$ m and $10\,000$ m from each other, respectively. 
The total-field anomaly data were corrected from daytime variation and 
subtracted from the main geomagnetic field using the IGRF. 
The inclination, declination and intensity of the main geomanetic field at the study area, 
for the period of the survey, are $-37.05^{\circ}$, $-18.17^{\circ}$ and 
$\approx 22 \, 768 $ nT, respectively.
To isolate the target total-field anomaly, we have applied 
a regional separation using a second-order polynomial fit resulting in the residual data 
shown in Fig. \ref{fig:real_data}a. 
Fig. \ref{fig:real_data}b and \ref{fig:real_data}c show the geometric height
(referred to the WGS84 ellipsoid) and UTM horizontal coordinates of the data and the 
topography, respectively. For convenience, we have subtracted $ 800 $ m from the height 
values.

The Anit{\'a}polis alkaline-carbonatitic complex forms a circular concentric body 
($\approx 6$ km$^{2}$ in area) containing magnetite as part of its mineralogical 
composition. It intruded into the Late Proterozoic leucogranites of the Dom Feliciano 
mobile belt in the Early Cretaceous ($132$ Ma), apparently concomitant with the 
voluminous flood tholeiitic basalts of the Serra Geral Formation ($133$-$130$ Ma) 
at the southern side of the Paran{\'a} Basin \citep{gibson-1999, scheibe-etal2005}.
As pointed out by \citet{GOMES2018}, there is still some debate about the emplacement 
of the Anit{\'a}polis alkaline-carbonatitic complex. 
\citet{melcher-coutinho1966} pointed out the influence of N-S-trending faults. 
\citet{scheibe-etal2005} considered that it is roughly emplaced along the E-W Rio 
Uruguay Lineament. According to \citet{riccomini-etal2005}, the Anit{\'a}polis 
complex does not show a clear structural control. 

We set the total-magnetization direction of the interpretation model with
inclination $ I=-21^\circ $ and declination $ D=-11^\circ $. These values were estimated 
by \cite{reis-seg-2019} for the study area by using a method based on the 
equivalent-layer technique \cite[]{dampney1969,emilia1973}. We have verified this estimated 
direction using the reduction to the pole technique (not shown). 
This total-magnetization direction indicates the presence of remanent magnetization.
Laboratory measurements on rock samples obtained at the Jacupiranga complex,
another alkaline complex located northward of the study area, 
with the same age as the Anit{\'a}polis complex, 
show total-magnetization intensities values varying from approximately 
$0.01$ to $29.90$ A/m \citep[][ tb. 1]{valdivia-2009}.
We used these values as a priori information to constraint the 
total-magnetization intensity $m_{0}$ of the Anit{\'a}polis complex.

We used an interpretation model formed by $L = 6$ prisms, each one with 
$V = 20$ vertices defining their horizontal cross-sections.
We inverted the observed total-field anomaly (Fig. \ref{fig:real_data}a) for each 
pair of $m_0$ and $z_0$ shown in Fig. \ref{fig:real_map}, resulting in $100$ estimated 
models. 
For all models, we set an initial approximation $\hat{\mathbf{p}}_{(0)}$ with origin 
at $(x_0^k, y_0^k) = (6921, 688)$ km, constant radii $r_j^k = 700$ m for 
all vertices forming all prisms and the same constant thickness $dz = 900$ m.
We also set the same weights which were used in the synthetic tests, i.e., $\tilde{\alpha}_1 = 10^{-4}$, 
$\tilde{\alpha}_2 = 10^{-3}$, 
$\tilde{\alpha}_3 = 10^{-4}$, $\tilde{\alpha}_4 = 0$, $\tilde{\alpha}_5 = 0$, 
$\tilde{\alpha}_6 = 10^{-8}$, and $\tilde{\alpha}_7 = 10^{-5}$ (eq. \ref{eq:alphas}). 

Fig. \ref{fig:real_map} shows that, similarly to the previous synthetic test with the 
complex model (Fig. \ref{fig:complex_map}), there is a region containing 
candidate solutions producing small values for the goal function 
$\Gamma(\mathbf{p})$ (eq. \ref{eq:gamma}), with different values of $m_0$ and $z_0$.
The pink diamond in Fig. \ref{fig:real_map} represents the estimated model 
shown in Fig. \ref{fig:real_result2}. 
This model produces the smallest value for $ \Gamma(\mathbf{p}) $ (\ref{eq:gamma}),
it has a volume $ 9.96 $ km$ ^3 $, total thickness of $ 4\,596.55 $ m 
($ dz = 766.10 $), depth-to-the-top $z_0 = 20$ m and total-magnetization intensity 
$m_0 = 14.0$ A/m.
Fig. \ref{fig:real_result}a shows that this estimated model produces a reasonable data fit.
This estimated depth-to-the-top $z_20$ indicates a non-outcropping source, 
which is compatible with a priori information a the study area. 
We do not have evidences of an outcropping source for this anomaly, 
although there are outcropping intrusions in the area of the Anit{\'a}polis 
complex \cite[]{gibson-1999}.
The estimated total-magnetization intensity $ m_0 $ is also compatible with the 
available a priori information. It is within the range found by 
\citet{valdivia-2009} in the Jacupiranga complex.

The white diamond in Fig. \ref{fig:real_map} represents the alternative model 
show in Fig. \ref{fig:real_result}. This model is similar to that shown in 
Fig. \ref{fig:real_result2}. It has a volume $ 9.57 $ km$ ^3 $, total thickness 
$ 4\,419.34 $ m ($ dz = 736.56 $), depth-to-the-top $z_0 = 0$ m and 
total-magnetization intensity $m_0 = 14$ A/m. 
In comparison with the estimated model shown in Fig. \ref{fig:real_result2}, the 
alternative model shown in Fig. \ref{fig:real_result} has a shallower top and a 
very similar geometry.
Both estimated models show an northwest-southeast elongated body with accentuated dip 
along depth (Figs \ref{fig:real_result2} and \ref{fig:real_result}).
This is the same direction associated with the Serra Geral Lineament, 
crossing the study area, the Ponta Grossa Arch and the Torres syncline,
which are prominent structural features located, respectively, 
northward and southward of the study area \citep[e.g., ][ p. 535]{scheibe-etal2005}.
The total-magnetization intensities and depths-to-the-top estimates are compatible with 
the available a priori information and produce reasonable data fits.
Hence, both models represent the possible geometry of the Anit{\'a}polis complex. 