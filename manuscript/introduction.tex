\section{Introduction}

The interpretation of a 3D magnetic survey measured above the surface of the earth is an important challenge in exploration geophysics. Many authors developed different strategies to interpret magnetic data quantitatively. The nonuniqueness of the magnetic inversion is well-known so that several subsurface sources can reproduce the same magnetic dataset with the same accuracy. Necessarily, a priori information is required in the inversion process to overcome this difficulty. By introducing it, the family of mathematically acceptable models should decrease and be more coherent to the local geology.

There are groups of approaches that deal differently with such nonuniqueness of the magnetic inverse problem. The prior information available can determine a suitable approach depending on the desired outcome of inversion. The first group approximates the magnetic source by a geometrically simple causative body with a small number of parameters that define the geometry and the physical property. Commonly, this approach consists of a nonlinear optimization problem with a small family of possible solutions due to the very restrictive parametrization \cite[]{ballantyne-1980,bhattacharyya-1980,silva-1983}.

The second group composes the vast majority of magnetic inversion methods. These methods divide the subsurface in a grid of rectangular prisms that have magnetization direction aligned with the local main field of the earth \cite[]{cribb-1976,li_3-d_1996,pilkington_3-d_1997} . In these methods, the magnetic susceptibility is a parameter in the inversion, considered isotropic within each prism. Additionally, some of these methods allow a magnetization direction different from the local main field direction \cite[]{pignatelli-2006}. In this case, instead of estimating susceptibility, these methods estimate the total-magnetization intensity within each prism. The magnetic sources in the subsurface are imagined by the estimated susceptibility distribution or magnetization intensity supporting the geological interpretation. Theoretically, these methods are capable of recovering the geometry of complex sources. However, they are characterized by a high computational cost due to the solution of large linear systems. Another disadvantage of these methods is the necessity to impose prior information to overcome problems of nonuniqueness and instability due to a large number of parameters in the inversion.

The third group of total-field anomaly inversion estimates the geometries or boundaries of a magnetic source by assuming some knowledge about the physical property. The method proposed by \cite{wang_inversion_1990} estimates, in the frequency domain, the spacial position of vertices defining a polyhedron body.  Among the few studies in this group, \cite{wenbin-2017} have developed a multiple level-set method to estimate the shape of a 3D magnetic source. This method represents the geological structure by a set of causative bodies with uniform magnetic susceptibility. According to the authors, this method is only applicable to magnetic data produced by sources with weak induced magnetization and known magnetic susceptibility. However, these methods have a small number of parameters for inversion than the second group. In addition, the inherent nonuniqueness of the magnetic inversion for these methods is not severe in comparison to the other groups due to the flexibility of the parametrization. As noticeable, the lack of works on 3D magnetic inversion presents a great challenge in the area of the potential fields.

The main difficulty in magnetic inversion to estimate the shape of a 3D magnetic source is that prior information about the physical property and position of the body is necessary as input. If available, prior information can help the algorithm to deal with fewer parameters in the inversion process. In other words, the whole process would be faster and more suitable for the geological configuration. Moreover, the choice of the best solution would be easier due to the smaller subset of possible solutions. Consequently, the inverted boundaries of the body can delineate more realistically the magnetic source. We thus are challenged to develop an algorithm that requires the total-magnetization direction only and retrieves complex geometries for an isolated source.

We present an algorithm to estimate the geometry of an isolated magnetic 3D source by inverting total-field anomaly data with known total-magnetization direction. The prisms have the same depth extent and total-magnetization intensity that are a parameter and a constant in the inversion process, respectively. The method estimates the radii connecting an arbitrary origin and the vertices of the regular polygon that describes the horizontal cross-section of the prisms. Also, the algorithm estimates the Cartesian coordinates of the arbitrary origin of the prisms polygons. The magnetic source is approximated by a set of vertically juxtaposed polygonal prisms to estimate the shape of the source. The method is an extension for magnetic data of those presented by \cite{oliveirajr-etal2011} and \cite{oliveirajr-etal2015} for inversion of gravity and gravity-gradient data, respectively. However, differently from those authors, we estimate the thickness of all prisms defining the interpretation model. Additionally, we introduced a validation test using successive inversions for ranges of depth to the top and the total-magnetization intensity of the source and choosing them based on the lower values of the objective function. To obtain a stable solutions, we introduced a set of six smoothness constraints following the strategy of \cite{oliveirajr-etal2011} and a new minimum Euclidean constraint on the depth extent of the prisms. Ultimately, tests on synthetic total-field anomaly data and a field application on the Anitápolis alkaline complex, Santa Catarina, Brazil, support the efficiency of our method.