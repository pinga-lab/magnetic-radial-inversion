\section{Introduction}

The interpretation of total-field anomalies on the surface of the Earth is an 
important challenge in exploration geophysics due to the nonuniqueness of 3-D magnetic 
inversion. It is well-known that several magnetization distribution in subsurface 
can reproduce the same magnetic data with the same accuracy. 
To overcome this inherent ambiguity, a priori information need to be introduced 
for reducing the number of possible possible solutions that are coherent with the local geology.
There are basically three groups of 3-D magnetic inversion methods. The available 
a priori information determines the suitable approach for each case.

The first group of methods approximates the source by a geometrically 
simple causative body having its geometry defined by a small number of parameters 
\cite[e.g., ][]{ballantyne-1980,bhattacharyya-1980,silva-1983}. These methods 
estimate both the geometry and the physical property of the source by solving 
a nonlinear inverse problem. Due to the very restrictive parametrization, 
such methods usually do not have severe problems with ambiguity.

The second group is formed by the vast majority of methods. 
They approximate the subsurface by a grid of juxtaposed rectangular prisms having 
a constant total-magnetization direction. Some methods presume that a purely induced 
magnetization \cite[e.g., ][]{cribb-1976,li_3-d_1996,pilkington_3-d_1997} and the 
isotropic magnetic susceptibility of the prisms is the quantity estimated by solving 
a linear inverse problem. Different approaches have improved this inversion method to obtain focused images of the subsurface. For example, \cite{portniaguine_focusing_1999} and \cite{portniaguine_3d_2002} introduced the minimum gradient support to minimize the effect of strong variations and discontinuity on the parameters by inverting magnetic anomaly and any component of the total anomalous field. \cite{barbosa_interactive_2006} presented a method for inverting interfering magnetic anomalies by combining  features of the forward modeling (the interactivity) and traditional inversion (the automatic data fitting). Other studies introduced strategies to constraint the nonuniqueness and delineate the source \cite[]{tontini,pilkington_3d_2009,shamsipour_3d_2011,cella_inversion_2012,abedi-2015}. Exceptionally, some of these methods allowed different magnetization direction from the local main field one \cite[e.g., ][]{pignatelli-2006}. 
In this case, the parameters to be estimated are the total-magnetization intensities 
of the prisms. 
In all these methods, and thus, the geometries of the magnetic sources are indirectly retrived by interpreting the estimated total-magnetization intensity 
distribution. 
Theoretically, these inversion methods are capable of recovering the geometry of complex 
sources. However, they require a plethora of a priori information to overcome 
their nonuniqueness and instability due to the large number of parameters 
to be estimated. Additionally, they are characterized by a high 
computational cost associated due to the solution of large linear systems.

The third group of 3-D magnetic inversion methods presume some knowledge about the 
physical property distribution and estimate the geometry of the sources. 
They are usually formulated as nonlinear inverse problems. 
\cite{wang_inversion_1990} approximate the source by a polyhedron and estimate 
the position of its vertices in the Fourier domain. 
\cite{wenbin-2017} have developed a multiple level-set method to estimate geometry 
of a set of causative bodies with uniform magnetic susceptibility. \cite{hidalgo-2019} inverted the total-field anomaly for estimating the geometry of a basement relief of a sedimentary basin with known magnetization intensity but unknown magnetization direction. 
This method has a small number of parameters to be estimated by inversion and 
has much less ambiguity in comparison to the second group. 

We present a magnetic inversion to estimate the geometry of an isolated 
and uniformly 3-D magnetic source having known total-magnetization direction.
Our method is an extension for total-field anomaly data of those methods presented 
by \cite{oliveirajr-etal2011} and \cite{oliveirajr-barbosa2013} for inverting 
gravity and gravity-gradient data, respectively. 
We approximate the source by an interpretation model formed by vertically juxtaposed 
right prisms having horizontal cross-sections defined by polygons, all
of them with the same number of vertices.
For convenience, all prisms have the same thickness and total-magnetization 
intensity.
Differently from \cite{oliveirajr-etal2011} and \cite{oliveirajr-barbosa2013}, 
our method estimates not only the horizontal Cartesian coordinates of the origins and the radii of the vertices describing the horizontal cross-sections of all prisms, but also the thickness of all 
prisms forming the interpretation model. Additionally, we perform a numerical analysis to investigate the sensitivity of our method to the use of different values of total-magnetization intensities and depths to the top 
of the shallowest prism. Among the estimated models, those producing the 
lowest values of goal function form the set of candidate models.
To obtain a stable solutions, we use the same set of regularizing functions proposed by 
\cite{oliveirajr-etal2011} and also propose a new one for constraining the 
thickness of the prisms. 

A test with synthetic data produced by a simple symmetric model shows the performance 
of our method in an ideal case. We also applied our method to interpret the synthetic 
data produced by a complex sources having variable shape and dip along depth. The 
results show that our method can be a very useful tool for interpreting magnetic data 
in real situations. Based on these synthetic tests, we applied our method 
to interpret airborne data over the alkaline-carbonatitic complex of 
Anit{\'a}polis, in the Santa Catarina state, in southern Brazil. 
Our results bring some light on the debate about the structural control of the 
Anit{\'a}polis complex. 
They suggest that it is controlled by a nearly N30W-trending fault
and has maximum bottom depth $\approx 4.5 $ km, in agreement with the available 
geological information at the study area.