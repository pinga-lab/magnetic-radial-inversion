\begin{summary}
We present a method for inverting total-field anomaly data to estimate the geometry of 
a uniformly magnetized 3-D geological source in the subsurface. The method assumes 
the total-magnetization direction is known. 
We approximate the source by an ensemble of vertically juxtaposed right prisms, all of them with the same total-magnetization vector and depth extent. 
The horizontal cross-section of each prism is defined by a polygon having the same number of vertices equally spaced from $0^{\circ}$ to $360^{\circ}$,  whose polygon vertices  
are described by polar coordinates with an origin defined by a horizontal location 
over the top of each prism. 
Because our method estimates the radii of each polygon vertex  we refer to it as 
\textit{radial inversion}.
The position of these vertices, the horizontal location of each prism and the depth extent of all prisms are the parameters to be estimated by solving a constrained nonlinear inverse problem of minimizing a goal function. 
We run successive inversions for a range of tentative total-magnetization intensities 
and depths to the top of the shallowest prism. The estimated models producing 
the lowest values of the goal function form the set of candidate solutions.
To obtain stabilized solutions, we impose the zeroth- and first-order Tikhonov 
regularizations on the shape of the prisms. The method allows estimating the geometry 
of both vertical and inclined sources, with a constant direction of magnetization, 
by using the Tikhonov regularization. 
Tests with synthetic data show that the method can be of utility in estimating the shape of the magnetic source even in the presence of a strong regional field.
Results obtained by inverting airborne total-field anomaly data over the 
Anit{\'a}polis alkaline-carbonatitic complex, in southern Brazil, 
suggest that the emplacement of the magnetic sources was controlled by NW-SE-trending 
faults at depth, in accordance with known structural features at the study area.
\end{summary}