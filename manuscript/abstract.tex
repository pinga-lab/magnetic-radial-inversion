\begin{summary}
We present a method for inverting total-field anomaly data to estimate the geometry of a 3-D complex geological source in subsurface. The method assumes magnetization vector and the depth to the top both known. We use an ensemble of vertically juxtaposed 3-D right prisms to approximate the shape of the geological source. Each prism has a homogeneous known magnetization and an unknown polygon as its horizontal cross-sections. The vertices of the polygons approximately describe the edges of horizontal depth slices of the source. All prisms’ polygons have the same fixed number of vertices, which are equally spaced with central angles from $0^{\circ}$ to $360^{\circ}$ and horizontally described by polar coordinates associated to an arbitrary origin within each prism. The method estimates the radii of all vertices, the horizontal Cartesian coordinates of all arbitrary origins, and the depth extent of all prism defining the shape of the interpretation model. We impose zeroth- and first-order Tikhonov regularizations as constraints on the shape of the estimated model to stabilize the inverse problem. The method allows estimating both vertical and inclined sources by a suitable use of first-order Tikhonov regularization. This regularization can be applied on either all or few parameters excluding the depth extent of the prisms. The tests on synthetic total-field anomaly data show efficiency of the method on retrieving the shape of the source for either dipping
\end{summary}